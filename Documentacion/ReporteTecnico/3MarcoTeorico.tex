%=========================================================
\chapter{Marco teórico}

\section{Ingeniería de software}


El software se ha incrustado profundamente en casi todos los aspectos de nuestras vidas y, como consecuencia, el número de personas que tienen interés en las características y funciones que brinda una aplicación específica ha crecido en forma notable, por lo que debe hacerse un esfuerzo concertado para entender el problema antes de desarrollar una aplicación de software, por otro lado, los requerimientos de la tecnología de la información que demandan los individuos, negocios y gobiernos se hacen más complejos con cada año que pasa. En la actualidad, grandes equipos de personas crean programas de cómputo que antes eran elaborados por un solo individuo. El software sofisticado, que alguna vez se implementó en un ambiente de cómputo predecible y autocontenido, hoy en día se halla incrustado en el interior de todo, desde la electrónica de consumo hasta dispositivos médicos o sistemas de armamento. La complejidad de estos nuevos sistemas y productos basados en computadora demanda atención cuidadosa a las interacciones de todos los elementos del sistema.

Hacer ingeniería con el software en todas sus formas y a través de todos sus dominios de aplicación, se ha convertido en una necesidad tangible . 

El análisis es el proceso clave de la construcción de modernas aplicaciones de sistemas de información y la base para el diseño y desarrollo de la aplicación de software robusto y vigoroso. 