%=========================================================
\chapter{Marco teórico}

\section{Ingeniería de software}

El software se ha incrustado profundamente en casi todos los aspectos de nuestras vidas y, como consecuencia, el número de personas que tienen interés en las características y funciones que brinda una aplicación específica ha crecido en forma notable, por lo que debe hacerse un esfuerzo concertado para entender el problema antes de desarrollar una aplicación de software, por otro lado, los requerimientos de la tecnología de la información que demandan los individuos, negocios y gobiernos se hacen más complejos con cada año que pasa. En la actualidad, grandes equipos de personas crean programas de cómputo que antes eran elaborados por un solo individuo. El software sofisticado, que alguna vez se implementó en un ambiente de cómputo predecible y autocontenido, hoy en día se halla incrustado en el interior de todo, desde la electrónica de consumo hasta dispositivos médicos o sistemas de armamento. La complejidad de estos nuevos sistemas y productos basados en computadora demanda atención cuidadosa a las interacciones de todos los elementos del sistema.

Hacer ingeniería con el software en todas sus formas y a través de todos sus dominios de aplicación, se ha convertido en una necesidad tangible . 

El análisis es el proceso clave de la construcción de modernas aplicaciones de sistemas de información y la base para el diseño y desarrollo de la aplicación de software robusto y vigoroso. 


Una de las primeras definiciones de ingenieía de software fue dada por Fritz Bauer en el año de
1969, quien define que la ingeniería de software es “el establecimiento y uso de principios robustos, orientados a obtener software económico que sea fiable y funcione de manera eficiente sobre máquinas
reales”\hyperlink{b07}{[7]}., es decir, el desarrollo de software debe seguir el proceso de la ingeniería.

Todo proceso tiene diferentes etapas, como se muestra en la figura 2.1, para el desarrollo de software existen cuatro capas: herramientas, métodos, procesos y compromiso con la calidad. Cada una de ellas es importante, sin embargo, la fundamental para el desarrollo de software es la capa de proceso, ya que es donde se define la estructura básica del producto hasta la culminación del mismo.

\subsection{Proceso de desarrollo de software}
El proceso de desarrollo de software es un conjunto de actividades que tienen como fin la producci´on
de un software. Este proceso puede tener diferentes actividades o etapas pero, fundamentalmente, debe
contener las siguientes:
1. An´alisis
2. Dise˜no
3. Implementaci´on
4. Pruebas
Dentro de estas cuatro actividades, se realizan diferentes tareas, por ejemplo, para la etapa de
an´alisis se realiza el documento de an´alisis en donde se describe el funcionamiento del sistema; durante
la etapa de dise˜no se generan los diagramas que describen el funcionamiento del sistema; en la fase
de implementaci´on se genera el c´odigo del software; y en la etapa de pruebas se valida y verifica el
software, sin embargo, dentro del desarrollo de estas cuatro etapas, existen actividades que se realizan
durante cualquiera de ellas. Estas actividades sombrilla sirven para la monitorizaci´on del proyecto.
Una de estas actividades es conocida como revisi´on t´ecnica formal, en donde todos los miembros del
equipo se re´unen para dar conocimiento de los avances que se tienen dentro del proyecto y, as´ı, evitar
errores a futuro.
Existen diferentes procesos de software, mejor conocidos como metodolog´ıa de desarrollo, en donde
podemos encontrar las ´agiles o las comunes, sin embargo, todas ellas siempre contendr´an dentro de su
proceso, al menos, las cuatro actividades ya mencionadas.