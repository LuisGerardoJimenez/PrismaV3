%=========================================================
\chapter{Estado del Arte}


	Es común que dentro del area de la ingenieria del software, se confundan los términos: ''Caso de uso'' y ''Diagrama de caso de uso'', sin embargo es importante resaltar las diferencias entre ellos para comprender el objetivo principal del proyecto terminal.
	
	 \begin{itemize}
		\item Un caso de uso es aquel que describe la interacción entre un actor y un sistema en forma de secuencia de acciones y circunstancias específicas.
		
		\item En cambio, un diagrama de casos de uso es el modelo de un grafo con dos tipos de nodos (Actor y caso de uso), el cual ilustra el comportamiento del caso de uso.
	\end{itemize}

	Por lo tanto, el diagrama no es más que una representación gráfica del caso de uso, un diagrama de casos de uso no  reemplaza el documento de casos de uso. En el diagrama no se describe la interacción detallada del sistema con los actores.\\
	

	Ahora bien, en la red hay una gran variedad de sistemas que permiten la generación de diagramas de casos de uso en UML,a partir del procesamiento del lenguaje natural sin embargo no hay herramientas comerciales o gratuitas que posibiliten la generación del documento con las especificaciones y la gestión de sus componentes.
	

%---------------------------------------------------------
\section{Antecedentes}

\subsection{UCD-Generator - Una aplicación LESSA para el diseño de casos de uso}

Las herramientas CASE convencionales requieren una comprensión completa del negocio, una gran cantidad de tiempo y esfuerzos adicionales por parte del analista del sistema durante el proceso de creación, organización, etiquetado y finalización de los diagramas de casos de uso. Es por esto que se diseñó un sistema que proporciona una manera rápida y confiable de generar diagramas de casos de uso para ahorrar tiempo y presupuesto tanto para el usuario como para el analista del sistema.

\subsubsection{Objetivo}
Este sistema presenta un enfoque basado en el procesamiento del lenguaje natural LESSA (Language Engineering System for semantic analysis) que se utiliza para comprender automáticamente el texto en lenguaje natural y extraer la información requerida. Esta información se utiliza para dibujar los diagramas de casos de uso. El usuario escribe sus preferencias basadas en la interfaz en inglés, en unos pocos párrafos y el sistema diseñado tiene una capacidad notable para analizar el script dado. Después del análisis compuesto y la extracción de información asociada, el sistema diseñado en realidad dibuja los diagramas de casos de uso. 

\subsection{Generación automatizada de diagramas de casos de uso a partir de marcos de problemas mediante el análisis de conceptos formales}




%
%\begin{table}[htbp!]
%	\begin{requerimientosU}
%		\FRitem{RU1}{Control de vehículos}{El usuario requiere llevar un registro actualizado de los vehículos, sus características y su estado.}{1}{\DONE}
%		\FRitem{RU2}{Registro de ventas}{El usuario requiere llevar un registro actualizado de todas las ventas realizadas por mes y su status: pedido, entregado, pagado, etc..}{2}{\TODO}
%		\FRitem{RU3}{Registro de clientes}{El usuario requiere llevar un registro actualizado de todos los clientes para su seguimiento, atención y tareas de promoción y mercadotecnia.}{1}{\DONE}
%		\FRitem{RU4}{Planeación de entregas}{El usuario requiere una herramienta que le facilite la planeación de vehículos para que esta sea la más adecuada.}{-}{\DOING}
%		\FRitem{...}{...}{...}{...}{...}
%	\end{requerimientosU}
%    \caption{Requerimientos funcionales del sistema.}
%    {\footnotesize\em Para leer correctamente esta tabla vea la leyenda en la Tabla~\ref{tbl:leyendaRF} en la página~\pageref{tbl:leyendaRF}.}
%    \label{tbl:reqFunc}
%\end{table}
%



