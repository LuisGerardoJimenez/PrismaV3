	\begin{UseCase}{CU5.1}{Registrar Módulo}{
		Este caso de uso permite al actor registrar la información de un módulo de un proyecto.
	}
		\UCitem{Versión}{\color{Gray}0.1}
		\UCitem{Actor}{\hyperlink{jefe}{Líder de Análisis}, \hyperlink{analista}{Analista}}
		\UCitem{Propósito}{Registrar la información de un módulo.}
		\UCitem{Entradas}{
		\begin{itemize}
			\item \cdtRef{moduloEntidad:claveModulo}{Clave:} Se escribe desde el teclado.
			\item \cdtRef{moduloEntidad:nombreModulo}{Nombre:} Se escribe desde el teclado.
			\item \cdtRef{moduloEntidad:descripcionModulo}{Descripción:} Se escribe desde el teclado.
		\end{itemize}	
		}
		\UCitem{Salidas}{\begin{itemize}
				\item \cdtRef{proyectoEntidad:claveProyecto}{Clave del proyecto:} Lo obtiene el sistema.
				\item \cdtRef{proyectoEntidad:nombreProyecto}{Nombre del proyecto:} Lo obtiene el sistema.
				\item \cdtIdRef{MSG1}{Operación exitosa}: Se muestra en la pantalla \IUref{IU4}{Gestionar Módulos} para indicar que el registro fue exitoso.
		\end{itemize}}
		\UCitem{Destino}{Pantalla}
		\UCitem{Precondiciones}{Ninguna}
		\UCitem{Postcondiciones}{
		\begin{itemize}
			\item Se registrará un módulo de un proyecto en el sistema.
			\item Se podrán gestionar los casos de uso del módulo.
			\item Se podrán gestionar las pantallas del módulo.
		\end{itemize}
		}
		\UCitem{Errores}{\begin{itemize}
		\item \cdtIdRef{MSG4}{Dato obligatorio}: Se muestra en la pantalla \IUref{IU4.1}{Registrar Módulo} cuando no se ha ingresado un dato marcado como obligatorio.
		\item \cdtIdRef{MSG29}{Formato incorrecto}: Se muestra en la pantalla \IUref{IU4.1}{Registrar Módulo} cuando el tipo de dato ingresado no cumple con el tipo de dato solicitado en
		el campo.
		\item \cdtIdRef{MSG6}{Longitud inválida}: Se muestra en la pantalla \IUref{IU4.1}{Registrar Módulo} cuando se ha excedido la longitud de alguno de los campos.
		\item \cdtIdRef{MSG7}{Registro repetido}: Se muestra en la pantalla \IUref{IU4.1}{Registrar Módulo} cuando se registre un módulo con un nombre o clave que ya se encuentra registrada en el sistema.
		\item \cdtIdRef{MSG18}{Caracteres inválidos}: Se muestra en la pantalla \IUref{IU4.1}{Registrar Módulo} cuando el actor ingrese una clave inválida, con base en la regla de negocio \BRref{RN2}{Nombres de los elementos}.
		\end{itemize}
		}
		\UCitem{Tipo}{Secundario, extiende del caso de uso \UCref{CU5}{Gestionar Módulos}.}
	\end{UseCase}
%--------------------------------------
	\begin{UCtrayectoria}
		\UCpaso[\UCactor] Solicita registrar un módulo oprimiendo el botón \IUbutton{Registrar} de la pantalla \IUref{IU4}{Gestionar Módulos}.
		\UCpaso[\UCsist] Muestra la pantalla \IUref{IU4.1}{Registrar Módulo}.
		\UCpaso[\UCactor] Ingresa la información solicitada en la pantalla. \label{CU5.1-P3}
		\UCpaso[\UCactor] Solicita guardar la información del módulo oprimiendo el botón \IUbutton{Aceptar} de la pantalla \IUref{IU4.1}{Registrar Módulo}. \Trayref{RM-A}
		\UCpaso[\UCsist] Verifica que el actor ingrese todos los campos obligatorios con base en la regla de negocio \BRref{RN8}{Datos obligatorios}. \Trayref{RM-B}
		\UCpaso[\UCsist] Verifica que los datos requeridos sean proporcionados correctamente con base en la regla de negocio \BRref{RN7}{Información correcta}. \Trayref{RM-C} \Trayref{RM-D}
		\UCpaso[\UCsist] Verifica que la clave y el nombre del módulo no se encuentre registrado en el sistema con base en la regla de negocio \BRref{RN6}{Unicidad de nombres}. \Trayref{RM-E}
		\UCpaso[\UCsist] Registra la información del módulo en el sistema.
		\UCpaso[\UCsist] Muestra el mensaje \cdtIdRef{MSG1}{Operación exitosa} en la pantalla \IUref{IU4}{Gestionar Módulos} para indicar al actor que el registro se ha realizado exitosamente.
	\end{UCtrayectoria}		
%--------------------------------------
	
	\begin{UCtrayectoriaA}{RM-A}{El actor desea cancelar la operación.}
		\UCpaso[\UCactor] Solicita cancelar la operación oprimiendo el botón \IUbutton{Cancelar} de la pantalla \IUref{IU4.1}{Registrar Módulo}
		\UCpaso[\UCsist] Muestra la pantalla \IUref{IU4}{Gestionar Módulos}.
	\end{UCtrayectoriaA}

	\begin{UCtrayectoriaA}{RM-B}{El actor no ingresó algún dato marcado como obligatorio.}
		\UCpaso[\UCsist] Muestra el mensaje \cdtIdRef{MSG4}{Dato obligatorio} y señala el campo que presenta el error en la pantalla \IUref{IU4.1}{Registrar Módulo}, indicando al actor que el dato es obligatorio.
		\UCpaso Regresa al paso \ref{CU5.1-P3} de la trayectoria principal.
	\end{UCtrayectoriaA}

	\begin{UCtrayectoriaA}{RM-C}{El actor proporciona un dato que excede la longitud máxima.}
		\UCpaso[\UCsist] Muestra el mensaje \cdtIdRef{MSG6}{Longitud inválida} y señala el campo que excede la longitud en la pantalla \IUref{IU4.1}{Registrar Módulo}, para indicar que el dato excede el tamaño máximo permitido.
		\UCpaso Regresa al paso \ref{CU5.1-P3} de la trayectoria principal.
	\end{UCtrayectoriaA}
	
	\begin{UCtrayectoriaA}{RM-D}{El actor ingresó un tipo de dato incorrecto.}
		\UCpaso[\UCsist] Muestra el mensaje \cdtIdRef{MSG29}{Formato incorrecto} y señala el campo que presenta el dato inválido en la pantalla \IUref{IU4.1}{Registrar Módulo}, para indicar que se ha ingresado un tipo de dato inválido.
		\UCpaso Regresa al paso \ref{CU5.1-P3} de la trayectoria principal.
	\end{UCtrayectoriaA}
	
	\begin{UCtrayectoriaA}{RM-E}{El actor ingresó una clave o nombre de módulo repetido.}
		\UCpaso[\UCsist] Muestra el mensaje \cdtIdRef{MSG7}{Registro repetido} y señala el campo que presenta la duplicidad en la pantalla \IUref{IU4.1}{Registrar Módulo}, indicando al actor que existe un módulo con el mismo nombre o clave.
		\UCpaso Regresa al paso \ref{CU5.1-P3} de la trayectoria principal.
	\end{UCtrayectoriaA}

	

