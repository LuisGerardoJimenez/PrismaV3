	\begin{UseCase}{CU9.1}{Registrar Mensaje}{
		Este caso de uso permite al analista registrar la información de un mensaje.
	}
		\UCitem{Versión}{\color{Gray}0.1}
		\UCitem{Actor}{\hyperlink{jefe}{Líder de Análisis}, \hyperlink{analista}{Analista}}
		\UCitem{Propósito}{Registrar la información de mensaje.}
		\UCitem{Entradas}{
		\begin{itemize}
			\item \cdtRef{MSGEntidad:numeroMSG}{Número:} Se escribe desde el teclado.
			\item \cdtRef{MSGEntidad:nombreMSG}{Nombre:} Se escribe desde el teclado.
			\item \cdtRef{MSGEntidad:descripcionMSG}{Nombre:} Se escribe desde el teclado.
			\item \cdtRef{MSGEntidad:redaccionMSG}{Redacción:} Se escribe desde el teclado.
			\item \cdtRef{MSGEntidad:paramtrizadoMSG}{Parametrizado:} Se escribe desde el teclado.
			\item Parámetros: Se escribe desde el teclado.
		\end{itemize}	
		}
		\UCitem{Salidas}{\begin{itemize}
				\item \cdtRef{MSGEntidad:claveMSG}{Clave:} Lo calcula el sistema mediante la regla de negocio \BRref{RN12}{Identifcador de elemento}.
				\item \cdtIdRef{MSG1}{Operación exitosa}: Se muestra en la pantalla \IUref{IU10}{Gestionar Mensajes} para indicar que el registro fue exitoso.
		\end{itemize}}
		\UCitem{Destino}{Pantalla}
		\UCitem{Precondiciones}{Ninguna}
		\UCitem{Postcondiciones}{
		\begin{itemize}
			\item Se registrará un mensaje en el sistema.
			\item El mensaje podrá ser referenciado en casos de uso.
		\end{itemize}
		}
		\UCitem{Errores}{\begin{itemize}
		\item \cdtIdRef{MSG4}{Dato obligatorio}: Se muestra en la pantalla \IUref{IU10.1}{Registrar Mensaje} cuando no se ha ingresado un dato marcado como obligatorio.
		\item \cdtIdRef{MSG29}{Formato incorrecto}: Se muestra en la pantalla \IUref{IU10.1}{Registrar Mensaje} cuando el tipo de dato ingresado no cumple con el tipo de dato solicitado en el campo.
		\item \cdtIdRef{MSG5}{Formato de campo incorrecto}: Se muestra en la pantalla \IUref{IU10.1}{Registrar Mensaje} cuando el número de mensaje contiene un carácter no válido.
		\item \cdtIdRef{MSG6}{Longitud inválida}: Se muestra en la pantalla \IUref{IU10.1}{Registrar Mensaje} cuando se ha excedido la longitud de alguno de los campos.
		\item \cdtIdRef{MSG7}{Registro repetido}: Se muestra en la pantalla \IUref{IU10.1}{Registrar Mensaje} cuando se registre un mensaje con un nombre o número que ya se encuentre registrado en el sistema.
		\item \cdtIdRef{MSG18}{Caracteres inválidos}: Se muestra en la pantalla \IUref{IU10.1}{Registrar Mensaje} cuando el nombre demensaje contiene un carácter no válido
		\end{itemize}.
		}
		\UCitem{Tipo}{Secundario, extiende del caso de uso \UCref{CU9}{Gestionar Mensajes}.}
	\end{UseCase}
%--------------------------------------
	\begin{UCtrayectoria}
		\UCpaso[\UCactor] Solicita registrar una mensaje oprimiendo el botón \IUbutton{Registrar} de la pantalla \IUref{IU10}{Gestionar Mensajes}.
		\UCpaso[\UCsist] Muestra la pantalla \IUref{IU10.1}{Registrar Mensaje: No parametrizado}.
		\UCpaso[\UCactor] Ingresa la información solicitada en la pantalla. \label{CU9.1-P3} \Trayref{RMSG-A}
		\UCpaso[\UCactor] Solicita guardar la información del mensaje oprimiendo el botón \IUbutton{Aceptar} de la pantalla \IUref{IU10.1}{Registrar Mensaje}. \label{CU9.1-P4} \Trayref{RMSG-B} 
		\UCpaso[\UCsist] Verifica que el actor ingrese todos los campos obligatorios con base en la regla de negocio \BRref{RN8}{Datos obligatorios}. \Trayref{RMSG-C}
		\UCpaso[\UCsist] Verifica que los datos requeridos sean proporcionados correctamente con base en la regla de negocio \BRref{RN7}{Información correcta}. \Trayref{RMSG-D} \Trayref{E}
		\UCpaso[\UCsist] Verifica que el número del mensaje sea proporcionado correctamente con base en la regla de negocio \BRref{RN7}{Información correcta}. \Trayref{RMSG-F}
		\UCpaso[\UCsist] Verifica que el número del mensaje no se encuentre registrado en el sistema con base en la regla de negocio \BRref{RN1}{Unicidad de números}. \Trayref{RMSG-G}
		\UCpaso[\UCsist] Verifica que el nombre de la mensaje no se encuentre registrado en el sistema con base en la regla de negocio \BRref{RN6}{Unicidad de nombres}. \Trayref{RMSG-H} 
		\UCpaso[\UCsist] Registra la información del mensaje en el sistema.
		\UCpaso[\UCsist] Muestra el mensaje \cdtIdRef{MSG1}{Operación exitosa} en la pantalla \IUref{IU10}{Gestionar Mensajes} para indicar al actor que el registro se ha realizado exitosamente.
	\end{UCtrayectoria}		
%--------------------------------------
	
	\begin{UCtrayectoriaA}{RMSG-A}{El actor desea ingresar un parámetro.}
		\UCpaso[\UCactor] Ingresa el token ''PARAM·''.
		\UCpaso[\UCsist] Muestra la pantalla \IUref{IU10.1A}{Registrar Mensaje: Parametrizado}.
		\UCpaso[\UCactor] Ingresa la descripción de cada parámetro.
		\UCpaso[\UCsist] Continúa en el paso \ref{CU9.1-P4} de la trayectoria principal.
	\end{UCtrayectoriaA}


	\begin{UCtrayectoriaA}{RMSG-B}{El actor desea cancelar la operación.}
		\UCpaso[\UCactor] Solicita cancelar la operación oprimiendo el botón \IUbutton{Cancelar} de la pantalla \IUref{IU10.1}{Registrar mensaje}.
		\UCpaso[\UCsist] Muestra la pantalla \IUref{IU10}{Gestionar Mensajes}.
	\end{UCtrayectoriaA}

	\begin{UCtrayectoriaA}{RMSG-C}{El actor no ingresó algún dato marcado como obligatorio.}
		\UCpaso[\UCsist] Muestra el mensaje \cdtIdRef{MSG4}{Dato obligatorio} y señala el campo que presenta el error en la pantalla \IUref{IU10.1}{Registrar Mensaje}, indicando al actor que el dato es obligatorio.
		\UCpaso Regresa al paso \ref{CU9.1-P3} de la trayectoria principal.
	\end{UCtrayectoriaA}

	\begin{UCtrayectoriaA}{RMSG-D}{El actor proporciona un dato que excede la longitud máxima.}
		\UCpaso[\UCsist] Muestra el mensaje \cdtIdRef{MSG6}{Longitud inválida} y señala el campo que excede la longitud en la pantalla \IUref{IU10.1}{Registrar Mensaje}, para indicar que el dato excede el tamaño máximo permitido.
		\UCpaso Regresa al paso \ref{CU9.1-P3} de la trayectoria principal.
	\end{UCtrayectoriaA}

	\begin{UCtrayectoriaA}{RMSG-E}{El actor ingresó un tipo de dato incorrecto.}
		\UCpaso[\UCsist] Muestra el mensaje \cdtIdRef{MSG29}{Formato incorrecto} y señala el campo que presenta el dato inválido en la pantalla \IUref{IU10.1}{Registrar Mensaje}, para indicar que se ha ingresado un tipo de dato inválido.
		\UCpaso Regresa al paso \ref{CU9.1-P3} de la trayectoria principal.
	\end{UCtrayectoriaA}

	\begin{UCtrayectoriaA}{RMSG-F}{El actor ingresó un número de mensaje con un tipo de dato incorrecto.}
		\UCpaso[\UCsist] Muestra el mensaje \cdtIdRef{MSG5}{Formato de campo incorrecto} y señala el campo que presenta el dato inválido en la pantalla \IUref{IU10.1}{Registrar Mensaje}, para indicar que se ha ingresado un tipo de dato inválido.
		\UCpaso Regresa al paso \ref{CU9.1-P3} de la trayectoria principal.
	\end{UCtrayectoriaA}

	\begin{UCtrayectoriaA}{RMSG-G}{El actor ingresó un número de mensaje repetido.}
		\UCpaso[\UCsist] Muestra el mensaje \cdtIdRef{MSG7}{Registro repetido} señala el campo que presenta la duplicidad en la pantalla \IUref{IU10.1}{Registrar Mensaje}, indicando al actor que existe un mensaje con el mismo número.
		\UCpaso Regresa al paso \ref{CU9.1-P3} de la trayectoria principal.
	\end{UCtrayectoriaA}
	
	\begin{UCtrayectoriaA}{RMSG-H}{El actor ingresó un nombre de un mensaje repetido.}
		\UCpaso[\UCsist] Muestra el mensaje \cdtIdRef{MSG7}{Registro repetido} y señala el campo que presenta la duplicidad en la pantalla \IUref{IU10.1}{Registrar Mensaje}, indicando al actor que existe un mensaje con el mismo nombre.
		\UCpaso Regresa al paso \ref{CU9.1-P3} de la trayectoria principal.
	\end{UCtrayectoriaA}
