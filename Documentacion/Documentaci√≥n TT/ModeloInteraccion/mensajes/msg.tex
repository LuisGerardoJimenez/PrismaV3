\section{Mensajes del sistema}

En esta sección se describen los mensajes utilizados en el prototipo actual del sistema. Los mensajes
se refieren a todos aquellos avisos que el sistema muestra al actor a través de la pantalla debido a diversas razones, por ejemplo: informar acerca de algún fallo en el sistema o para notificar acerca de alguna operación importante sobre la información.

\begin{mensaje}{MSG1}{Operación exitosa}{Notificación}
	\item [Objetivo:] Notificar al actor que la operación solicitada fue realizada exitosamente.
	\item[Redacción:] DETERMINADO ENTIDAD ha sido OPERACIÓN exitosamente.
	\item[Parámetros:] El mensaje se muestra con base en los siguientes parámetros:
		\begin{itemize}
 			\item DETERMINADO ENTIDAD: Artículo determinado más el nombre de la entidad sobre la que se realiza la operación.
 			\item OPERACIÓN: Es la acción que el actor solicitó realizar. Redactada en pasado.
		\end{itemize}
	\item[Ejemplo:] \begin{itemize}
		\item El {\em actor} ha sido {\em registrado} exitosamente.
		\item El {\em Caso de uso} ha sido {\em modificado} exitosamente.
	\end{itemize}
\end{mensaje}
\begin{mensaje}{MSG2}{No existe información}{Notificación}
	\item [Objetivo:] Notificar al actor que aún no existe información registrada en el editor.
	\item[Redacción:] No se han encontrado registros.
\end{mensaje}
\begin{mensaje}{MSG3}{Caso de uso terminado}{Notificación}
	\item [Objetivo:] Notificar al actor que ha terminado de registrar el caso de uso y este está listo para ser revisado.
	\item[Redacción:] El caso de uso ha sido registrado exitosamente. Ahora el caso de uso puede ser revisado
\end{mensaje}
\begin{mensaje}{MSG4}{Dato Obligatorio}{Error}
	\item [Objetivo:] Notificar al actor la omisión de algún dato obligatorio por ingresar
	\item[Redacción:] Campo Obligatorio.
\end{mensaje}
\begin{mensaje}{MSG5}{Formato de campo incorrecto}{Error}
	\item [Objetivo:] Notificar al actor que el dato no tiene el formato solicitado.
	\item[Redacción:] Formato incorrecto, ingrese ARTICULO TIPODATO.
	\item[Parámetros:] El mensaje se muestra con base en los siguientes parámetros:
		\begin{itemize}
 			\item TIPODATO: Indica el tipo de dato, por ejemplo cadena o número.
 			\item ARTICULO: Es el artículo que refiere gramaticalmente el género de un tipo de dato.
		\end{itemize}
	\item[Ejemplo:] \begin{itemize}
		\item Dato incorrecto, ingrese una {\em cadena}.
		\item Dato incorrecto, ingrese un {\em número}.
	\end{itemize}
\end{mensaje}
\begin{mensaje}{MSG6}{Longitud inválida}{Error}
	\item [Objetivo:] Notificar al actor que el dato ingresado en alguno de los campos del formulario no cumple con la longitud especificada.
	\item[Redacción:] Escriba menos de TAMAñoO TIPODATO.
	\item[Parámetros:] El mesaje se muestra con base en los siguientes parámetros:
		\begin{itemize}
 			\item TAMAÑO: Indica el tamaño requerido del campo.
 			\item TIPODATO: Inidica la unidad en que se mide la longitud del campo.
		\end{itemize}
	\item[Ejemplo:] 
	\begin{itemize}
		\item Escriba menos de {\em 30 letras}.
	\end{itemize}
\end{mensaje}
\begin{mensaje}{MSG7}{Registro duplicado}{Error}
	\item [Objetivo:] Notificar al actor que ya existe un elemento con las mismas características.
	\item[Redacción:] DETERMINADO ENTIDAD VALOR ya existe.
	\item[Parámetros:] El mensaje se muestra con base en los siguientes parámetros:
		\begin{itemize}
 			\item DETERMINADO ENTIDAD: Es un artículo determinado más la entidad sobre la que se efectúa la operación.
 			\item VALOR: Es el valor que toma determinado atributo de la entidad.
		\end{itemize}
	\item[Ejemplo:] \begin{itemize}
		\item El {\em Caso de uso 1} ya existe.
		\item El {\em Módulo admisión} ya existe.
	\end{itemize}
\end{mensaje}
\begin{mensaje}{MSG8}{Caso de uso terminado}{Confirmación}
	\item [Objetivo:] Preguntar al usuario si desea continuar con la operación debido a que no podrá realizar ningún cambio.
	\item[Redacción:] ''¿Está seguro de marcar como terminado este elemento? La información será enviada a revisión y no podrá realizar ningún cambio''.
\end{mensaje}
\begin{mensaje}{MSG9}{Elemento no referenciado}{Error}
	\item [Objetivo:] Notificar al actor que un actor, entrada, salida, regla de negocio o mensaje no está siendo utilizado en las trayectorias.
	\item[Redacción:] Hay elementos que no están siendo utilizados en las trayectorias.
\end{mensaje}
\begin{mensaje}{MSG10}{Confirmar eliminación}{Confirmación}
	\item [Objetivo:] Preguntar al actor si desea confirmar la eliminación.
	\item[Redacción:] ¿Está seguro de que quiere eliminar este registro?
\end{mensaje}
\begin{mensaje}{MSG11}{Elemento no agregado}{Error}
	\item [Objetivo:] Informar al actor que algunos elementos mencionados en las trayectorias no están en la descripción del caso de uso.
	\item[Redacción:] Hay elementos que están siendo utilizados en las trayectorias pero no están en la descripción del caso de uso.
\end{mensaje}
\begin{mensaje}{MSG12}{Ha ocurrido un error}{Error}
	\item [Objetivo:] Informar al actor que no es posible realizar la operación debido a que ha ocurrido un error inesperado en el sistema.
	\item[Redacción:] Ha ocurrido un error.
\end{mensaje}
\begin{mensaje}{MSG13}{Eliminación no permitida}{Error}
	\item [Objetivo:] Informar al actor que el elemento seleccionado no puede eliminarse debido a que está siendo referenciado en algún caso de uso.
	\item[Redacción:] Este elemento no se puede eliminar debido a que está siendo referenciado en: LISTA
	\item[Parámetros:] El mensaje se muestra con base en los siguientes parámetros:
	\begin{itemize}
		\item LISTA: Es la lista de razones por las que no se puede eliminar el elemento.
	\end{itemize}
	\item[Ejemplo:] \begin{itemize}
		\item Este elemento no se puede eliminar debido a que está siendo referenciado en: {\em Paso 14 de la trayectoria principal del caso de uso CUAD1.3.1 Agregar Horario de Entrevista}.
	\end{itemize}
\end{mensaje}
\begin{mensaje}{MSG14}{Dato no registrado}{Error}
	\item [Objetivo:] Informar al actor que el dato que ingresó o referenció no existe en el sistema.
	\item[Redacción:] Datos incorrectos, DETERMINADO ELEMENTO VALOR no se encuentra registrado en el sistema.
	\item[Parámetros:] El mensaje se muestra con base en los siguientes parámetros:
	\begin{itemize}
		\item DETERMINADO ENTIDAD: Artículo determinado más el nombre de un elemento.
		\item VALOR: El nombre o identificador de la entidad que no está registrada en el sistema.
	\end{itemize}
	\item[Ejemplo:] \begin{itemize}
		\item Dato incorrectos, el {\em actor Coordinador control escolar} no se encuentra registrado en el sistema.
	\end{itemize}
\end{mensaje}
\begin{mensaje}{MSG15}{Registro incorrecto}{Error}
	\item [Objetivo:] Informar al actor que alguno de los registros de alguna gestión no es correcto.
	\item[Redacción:] Alguno de DETERMINADO ENTIDAD no es correcto.
	\item[Parámetros:] El mensaje se muestra con base en los siguientes parámetros:
	\begin{itemize}
		\item DETERMINADO ENTIDAD: Artículo determinado más el nombre de un elemento.
	\end{itemize}
	\item[Ejemplo:] \begin{itemize}
		\item Alguna de los {\em mensajes} no es correcto.
	\end{itemize}
\end{mensaje}
\begin{mensaje}{MSG16}{Registro necesario}{Error}
	\item [Objetivo:] Informar al actor que debe realizar el registro de al menos un elemento para continuar con la operación.
	\item[Redacción:] Ingrese al menos INDETERMINADO ENTIDAD.
	\item[Parámetros:] El mensaje se muestra con base en los siguientes parámetros:
	\begin{itemize}
		\item INDETERMINADO ENTIDAD: Artículo indeterminado más el nombre de una entidad.
	\end{itemize}
	\item[Ejemplo:] \begin{itemize}
		\item Ingrese al menos un {\em paso}.
		\item Ingrese al menos un {\em caso de uso}.
	\end{itemize}
\end{mensaje}
\begin{mensaje}{MSG17}{Falta información}{Error}
	\item [Objetivo:] Informar al actor que es necesario que registre un elemento para solicitar la operación.
	\item[Redacción:] No es posible realizar la operación debido a que no ha registrado ELEMENTO.
	\item[Parámetros:] El mensaje se muestra con base en los siguientes parámetros:
	\begin{itemize}
		\item ELEMENTO: Elemento o elementos que son necesarios para solicitar la operación.
	\end{itemize}
	\item[Ejemplo:] \begin{itemize}
		\item No es posible realizar la operación debido a que no ha registrado {\em colaboradores}.
	\end{itemize}
\end{mensaje}
\begin{mensaje}{MSG18}{Caracteres inválidos}{Error}
	\item [Objetivo:] Informar al actor que el dato que ingresó no puede contener coma, punto, punto medio, dos puntos o guión bajo.
	\item[Redacción:] DETERMINADO ATRIBUTO no puede contener coma, punto, punto medio, dos puntos o guión bajo.
	\item[Parámetros:] El mensaje se muestra con base en los siguientes parámetros:
	\begin{itemize}
		\item DETERMINADO ATRIBUTO: Artículo determinado más el nombre del atributo que no puede contener coma, punto, punto medio, dos puntos o guión bajo.
	\end{itemize}
	\item[Ejemplo:] \begin{itemize}
		\item {\em El nombre} no puede contener coma, punto, punto medio, dos puntos o guión bajo.
	\end{itemize}
\end{mensaje}
\begin{mensaje}{MSG19}{Token incorrecto}{Error}
	\item [Objetivo:] Informar al actor que no es posible realizar la operación debido a que el token ingresado es incorrecto.
	\item[Redacción:] El token ingresado para DETERMINADO ELEMENTO es incorrecto.
	\item[Parámetros:] El mensaje se muestra con base en los siguientes parámetros:
	\begin{itemize}
		\item DETERMINADO ELEMENTO: Artículo determinado más el nombre del elemento que se desea referenciar.
	\end{itemize}
	\item[Ejemplo:] \begin{itemize}
		\item El token ingresado para el {\em caso de uso} es incorrecto.
	\end{itemize}
\end{mensaje}
\begin{mensaje}{MSG20}{Formato de archivo incorrecto}{Error}
	\item [Objetivo:] Informar al actor que el archivo seleccionado no cumple con el formato especificado en el modelo conceptual.
	\item[Redacción:] Formato incorrecto, seleccione un archivo con formato FORMATO.
	\item[Parámetros:] El mensaje se muestra con base en los siguientes parámetros:
	\begin{itemize}
		\item FORMATO: Es el formato o formatos que se permiten para el archivo de acuerdo al modelo conceptual.
	\end{itemize}
	\item[Ejemplo:] \begin{itemize}
		\item Formato incorrecto, seleccione un archivo con formato {\em jpeg}.
		\item Formato incorrecto, seleccione un archivo con formato {\em png}.
	\end{itemize}
\end{mensaje}
\begin{mensaje}{MSG21}{Se ha excedido el tamaño del archivo}{Error}
	\item [Objetivo:] Informar al actor que el archivo seleccionado excede el tamaño máximo especificado en el modelo conceptual.
	\item[Redacción:] El archivo no puede exceder TAMAÑO UNIDAD.
	\item[Parámetros:] El mensaje se muestra con base en los siguientes parámetros:
	\begin{itemize}
		\item TAMAÑO: Es el tamaño máximo permitido especificado en el modelo conceptual.
		\item UNIDAD: Es la unidad en que se especificó el tamaño máximo del archivo.
	\end{itemize}
	\item[Ejemplo:] \begin{itemize}
		\item El archivo no puede exceder {\em 2 MB}.
	\end{itemize}
\end{mensaje}
\begin{mensaje}{MSG22}{Modificación no permitida}{Error}
	\item [Objetivo:] Informar al actor que el elemento seleccionado no puede modificarse debido a que se encuentra asociado algún caso de uso liberados.
	\item[Redacción:] Este elemento no se puede modificar debido a que está siendo referenciado en casos de uso liberados: LISTA
	\item[Parámetros:] El mensaje se muestra con base en los siguientes parámetros:
	\begin{itemize}
		\item LISTA: Es la lista de casos de uso liberados en los que se utiliza el elemento.
	\end{itemize}
	\item[Ejemplo:] \begin{itemize}
		\item Este elemento no se puede modificar debido a que está siendo referenciado en casos de uso liberados: {\em CUAD1 Gestionar Entrevistadores para Oferta Académica, CUAD1.2 Agregar Programas Académicos Ofertados}.
	\end{itemize}
\end{mensaje}
\begin{mensaje}{MSG23}{Correo electrónico y/o contraseña incorrectos}{Error}
	\item [Objetivo:] Notificar al actor que no es posible iniciar sesión debido a que su correo y/o contraseña son incorrectos.
	\item[Redacción:] Correo electrónico y/o contraseña incorrectos.
\end{mensaje}
\begin{mensaje}{MSG24}{Recuperar contraseña}{Notificación}
	\item [Objetivo:] Informar al actor que se ha enviado un correo electrónico con su contraseña.
	\item[Redacción:] Se ha enviado un correo electrónico con su contraseña, favor de verificarlo.
\end{mensaje}
\begin{mensaje}{MSG25}{Datos de Sesión}{Notificación}
	\item [Objetivo:] Proporcionar al usuario sus datos para iniciar sesión.
	\item[Redacción:] Bienvenido(a) a TESSERACT, los datos con los que deberá iniciar sesión son: Nombre de usuario: NOMBRE, Contraseña: CONTRASEÑA.
	\item[Parámetros:] El mensaje se muestra con base en los siguientes parámetros:
	\begin{itemize}
		\item NOMBRE: gerardo@mail.com
		\item CONTRASEÑA: 123456
	\end{itemize}
\end{mensaje}
\begin{mensaje}{MSG26}{Orden de fechas}{Error}
	\item [Objetivo:] Informar al actor que las fechas de término deben ser posteriores a las fechas de inicio.
	\item[Redacción:] La FECHATERMINO debe ser posterior a la FECHAINICIO.
	\item[Parámetros:] El mensaje se muestra con base en los siguientes parámetros:
	\begin{itemize}
		\item FECHATERMINO: Es la fecha de término.
		\item FECHAINICIO: Es la fecha de inicio.
	\end{itemize}
\item[Ejemplo:] \begin{itemize}
	\item La { \em fecha de inicio programada} debe ser posterior a la {\em fecha de término programada}.
\end{itemize}
\end{mensaje}
\begin{mensaje}{MSG27}{Confirmación  de término}{Notificación}
	\item [Objetivo:] Solicitar la confirmación del actor para terminar el caso de uso.
	\item[Redacción:] ¿Está seguro de que quiere terminar el caso de uso?
\end{mensaje}
\begin{mensaje}{MSG28}{Longitud de CURP inválida}{Error}
	\item [Objetivo:] Informar al actor que la CURP ingresada no cumple con la longitud especificada.
	\item[Redacción:] La CURP debe contener exactamente 18 caracteres alfanuméricos.
\end{mensaje}
\begin{mensaje}{MSG29}{Formato incorrecto}{Error}
	\item [Objetivo:] Indicar al actor que el dato ingresado en alguno de los campos del formulario no cumple con el tipo de dato definido en el diccionario de datos.
	\item[Redacción:] Formato incorrecto.
\end{mensaje}


\section{Menús}

\subsection{Menú de Administrador}
 
 Este menú permite al actor {\em {\hyperlink{admin}{Administrador}}} seleccionar entre dos opciones:
 
 	\begin{enumerate}
 		\item Proyectos. Esta opción permite acceder a la gestión de proyectos.
 		\item Personal. Esta opción permite acceder a la gestión de personal.
 	\end{enumerate}
 
 \IUfig[1]{interfaces/menuAnalista}{MN1}{Menú de Administrador}
 
 \subsection{Menú de Colaborador} Este menú permite a los actores {\em {\hyperlink{lider}{Líder de proyecto}}} y {\em {\hyperlink{analista}{Analista}}} seleccionar la opción {\em Proyectos} la cual permite acceder a la gestión de proyectos dependiendo el rol.
 
  \IUfig[1]{interfaces/menuAnalista}{MN2}{Menú de Colaborador}