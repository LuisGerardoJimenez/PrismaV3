%--------------------------------------
\section{IU 6.1.4 Gestionar puntos de extensión}

\subsection{Objetivo}
	En esta pantalla el actor puede visualizar los puntos de extensión registrados y puede solicitar las operaciones de registrar, modificar y eliminar puntos de extensión.
\subsection{Diseño}
	En la figura \IUref{IU6.1.4}{Gestionar puntos de extensión} se muestra la pantalla ''Gestionar puntos de extensión'', por medio de la cual se podrán gestionar los puntos de extensión a través de una tabla. El actor podrá solicitar el registro, la modificación y la eliminación de un punto de extensión mediante los botones \IUbutton{Registrar}, \editar, \eliminar, respectivamente.
	
	En la parte superior derecha, el sistema muestra el proyecto, el módulo y el caso de uso en el que actualmente se encuentra trabajando.

\IUfig[1]{interfaces/IU6-1-4gestionarPuntosExt}{IU6.1.4}{Gestionar Puntos de Extensión}
\subsection{Comandos}
\begin{itemize}
	\item \IUbutton{Regresar}: Permite al actor regresar a la gestión de casos de uso, dirige a la pantalla \IUref{IU6}{Gestionar Casos de uso}
	\item \IUbutton{Registrar}: Permite al actor solicitar el registro de un punto de extensión, dirige a la pantalla \IUref{IU6.1.4.1}{Registrar punto de extensión}
	\item \editar [Modificar]: Permite al actor solicitar la modificación de un punto de extensión, \IUref{IU6.1.4.2}{Modificar punto de extensión}
	\item \eliminar [Eliminar]: Permite al actor solicitar la eliminación de un punto de extensión, dirige a una pantalla emergente.
\end{itemize}

\subsection{Mensajes}

\begin{Citemize}
	\item \cdtIdRef{MSG2}{No existe información}: Se muestra en la pantalla \IUref{IU6.1.1}{Gestionar Trayectorias} cuando no existen puntos de extensión registradas.
\end{Citemize}
