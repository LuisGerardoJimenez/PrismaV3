%--------------------------------------
\subsection{IU 9.1 Registrar Regla de Negocio}

\subsubsection{Objetivo}
	Esta pantalla permite al actor registrar la información de una regla de negocio nueva.
\subsubsection{Diseño}
	En la figura \IUref{IU9.1}{Registrar Regla de Negocio} se muestra la pantalla ''Registrar Regla de Negocio'' que permite registrar una regla de negocio. El actor deberá seleccionar el tipo de regla de negocio y el sistema mostrará los campos de los parámetros de la regla de negocio.
	
	Cuando el actor seleccione el tipo''Comparación de atributos'', el sistema mostrará la pantalla \IUref{IU9.1A}{Registrar regla de negocio: Comparación de atributos}; si el actor selecciona el tipo ''Unicidad de parámetros'' el sistema mostrará la pantalla \IUref{IU9.1B}{Registrar regla de negocio: Unicidad de parámetros}; si el actor selecciona el tipo ''Formato correcto'' el sistema mostrará la pantalla \IUref{IU9.1C}{Formato correcto}. Para los demás tipos de reglas de negocio el sistema no mostrará más campos.
	
	Una vez ingresada la información solicitada, el actor deberá oprimir el botón \IUbutton{Aceptar}. El sistema validará y registrará la información sólo si se han cumplido todas las reglas de negocio establecidas.
	
	Finalmente se mostrará el mensaje \cdtIdRef{MSG1}{Operación exitosa} en la pantalla \IUref{IU9}{Gestionar Reglas de Negocio}, para indicar que la información de la regla de negocio se ha registrado correctamente.
	En la parte superior derecha, el sistema muestra el proyecto en el que se encuentra trabajando.

\IUfig[1]{interfaces/IU9-1registrarBR}{IU9.1}{Registrar Regla de Negocio}
\IUfig[1]{interfaces/IU9-1AregistrarBR}{IU9.1A}{Registrar Regla de Negocio: Comparación de atributos}
\IUfig[1]{interfaces/IU9-1BregistrarBR}{IU9.1B}{Registrar Regla de Negocio: Unicidad de parámetros}
\IUfig[1]{interfaces/IU9-1CregistrarBR}{IU9.1C}{Registrar Regla de Negocio: Formato correcto}
\subsubsection{Comandos}
\begin{itemize}
	\item \IUbutton{Aceptar}: Permite al actor guardar el registro de una regla de negocio, dirige a la pantalla \IUref{IU9}{Gestionar Reglas de Negocio}.
	\item \IUbutton{Cancelar}: Permite al actor cancelar el registro de una regla de negocio, dirige a la pantalla \IUref{IU9}{Gestionar Reglas de Negocio}
\end{itemize}

\subsubsection{Mensajes}

\begin{Citemize}
	\item \cdtIdRef{MSG1}{Operación exitosa}: Se muestra en la pantalla \IUref{IU9}{Gestionar Reglas de Negocio} para indicar que el registro fue exitoso.
	\item \cdtIdRef{MSG4}{Dato obligatorio}: Se muestra en la pantalla \IUref{IU9.1}{Registrar Regla de Negocio} cuando no se ha ingresado un dato marcado como obligatorio.
	\item \cdtIdRef{MSG5}{Dato incorrecto}: Se muestra en la pantalla \IUref{IU9.1}{Registrar Regla de Negocio} cuando el tipo de dato ingresado no cumple con el tipo de dato solicitado en el campo.
	\item \cdtIdRef{MSG6}{Longitud inválida}: Se muestra en la pantalla \IUref{IU9.1}{Registrar Regla de Negocio} cuando se ha excedido la longitud de alguno de los campos.
	\item \cdtIdRef{MSG7}{Registro repetido}: Se muestra en la pantalla \IUref{IU9.1}{Registrar Regla de Negocio} cuando se registre una regla de negocio con un nombre o número que ya se encuentre registrado.
	\item \cdtIdRef{MSG18}{Caracteres inválidos}: Se muestra en la pantalla \IUref{IU9.1}{Registrar Regla de Negocio} cuando el nombre de la regla de negocio contiene un carácter no válido.
\end{Citemize}
