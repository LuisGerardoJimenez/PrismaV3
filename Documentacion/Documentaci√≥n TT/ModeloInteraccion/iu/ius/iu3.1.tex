%--------------------------------------
\section{IU 3.1 Registrar Persona}

\subsection{Objetivo}
	Esta pantalla permite al actor registrar la información de una persona.
\subsection{Diseño}
	En la figura \ref{IU3.1} se muestra la pantalla ''Registrar persona'' que permite al actor registrar una persona.
	Una vez ingresada la información solicitada, el actor deberá oprimir el botón \IUbutton{Aceptar} . El sistema validará y registrará la información solo si se han cumplido todas las reglas de negocio establecidas.
	
	Finalmente se mostrará el mensaje \cdtIdRef{MSG1}{Operación Exitosa} en la pantalla \IUref{IU3}{Gestionar Personal}, para indicar que la información de la persona se ha registrado correctamente.

\IUfig[1]{interfaces/IU3-1registrarPersona}{IU3.1}{Registrar Persona}\label{IU3.1}
\subsection{Comandos}
\begin{itemize}
	\item \IUbutton{Aceptar}: Permite al actor guardar el registro de una persona, dirige a la pantalla \IUref{IU3}{Gestionar Personal}
	\item \IUbutton{Cancelar}: Permite al actor cancelar el registro de una persona, dirige a la pantalla \IUref{IU3}{Gestionar Personal}
\end{itemize}

\subsection{Mensajes}

\begin{Citemize}
	\item \cdtIdRef{MSG1}{Operación exitosa}: Se muestra en la pantalla \IUref{IU3}{Gestionar Personal} para indicar que el registro fue exitoso.
	\item \cdtIdRef{MSG4}{Dato obligatorio}: Se muestra en la pantalla \IUref{IU3.1}{Registrar proyecto} cuando no se ha ingresado un dato marcado como obligatorio.
	\item \cdtIdRef{MSG5}{Dato incorrecto}: Se muestra en la pantalla \IUref{IU3.1}{Registrar persona} cuando el tipo de dato ingresado no cumple con el tipo de dato solicitado en el campo.
	\item \cdtIdRef{MSG6}{Longitud inválida}: Se muestra en la pantalla \IUref{IU3.1}{Registrar persona} cuando se ha excedido la longitud de alguno de los campos.
	\item \cdtIdRef{MSG7}{Registro repetido}: Se muestra en la pantalla \IUref{IU3.1}{Registrar persona} cuando se registre una persona con una CURP que ya exista.
\end{Citemize}
