%--------------------------------------
\subsection{IU 12.1.2 Modificar Atributo}

\subsubsection{Objetivo}
	Esta pantalla permite al actor modificar la información de un atributo.
\subsubsection{Diseño}
	En la figura \IUref{IU12.1.2}{Modificar Atributo} se muestra la pantalla ''Modificar Atributo'' que permite realizar cambios a un atributo perteneciente a una entidad de un proyecto.
	Una vez ingresada la información solicitada, el actor deberá oprimir el botón \IUbutton{Aceptar} . El sistema validará y registrará la información solo si se han cumplido todas las reglas de negocio establecidas.
	
	En la parte superior derecha, el sistema muestra el proyecto en el que se encuentra trabajando.
	
	\IUfig[1]{interfaces/IU12-1-2modificarAtributo}{IU12.1.2}{Modificar Atributo}
	
	La información mostrada dependerá del tipo de dato:
	\begin{itemize}
		\item Tipo de dato ''Archivo'' el sistema solicitará los formatos válidos así como el tamaño máximo, como se muestra en la figura \IUref{IU12.1.1A}{Registrar Atributo: Archivo}.
		\IUfig[1]{interfaces/IU12-1-2AmodificarAtributo}{IU12.1.2A}{Modificar Atributo: Archivo}
		\item Tipo de dato ''Booleano'' o ''Fecha'', el sistema no solicitará ningún dato extra, como se muestra en la figura \IUref{IU12.1.1B}{Registrar Atributo: Booleano, Fecha}
		\IUfig[1]{interfaces/IU12-1-2BmodificarAtributo}{IU12.1.2B}{Modificar Atributo: Booleano, Fecha}
		\item Tipo de dato ''Cadena" o ''Entero'' o ''Flotante'', el sistema solicitará la longitud máxima, como se muestra en la figura \IUref{IU12.1.1C}{Registrar Atributo: Cadena, Entero, Flotante}
		\IUfig[1]{interfaces/IU12-1-2CmodificarAtributo}{IU12.1.2C}{Modificar Atributo: Cadena, Entero, Flotante}
		\item Tipo de dato ''Otro'', el sistema solicitará el tipo especificado, como se muestra en la figura \IUref{IU12.1.1D}{Registrar Atributo: Otro}
		\IUfig[1]{interfaces/IU12-1-2DmodificarAtributo}{IU12.1.2D}{Modificar Atributo: Otro}
	\end{itemize}

\subsubsection{Comandos}
\begin{itemize}
	\item \IUbutton{Aceptar}: Permite al actor guardar los cambios del atributo, dirige a la pantalla \IUref{IU12.1}{Registrar Entidad} o \IUref{IU12.2}{Modificar Entidad}.
	\item \IUbutton{Cancelar}: Permite al actor cancelar la edición del atributo, dirige a la pantalla \IUref{IU12.1}{Registrar Entidad} o \IUref{IU12.2}{Modificar Entidad}.
\end{itemize}

\subsubsection{Mensajes}

\begin{Citemize}
	\item \cdtIdRef{MSG4}{Dato obligatorio}: Se muestra en la pantalla \IUref{IU12.1.2}{Modificar Atributo} cuando no se ha ingresado un dato marcado como obligatorio.
	\item \cdtIdRef{MSG5}{Dato incorrecto}: Se muestra en la pantalla \IUref{IU12.1.2}{Modificar Atributo} cuando el tipo de dato ingresado no cumple con el tipo de dato solicitado en el campo.
	\item \cdtIdRef{MSG6}{Longitud inválida}: Se muestra en la pantalla \IUref{IU12.1.2}{Modificar Atributo} cuando se ha excedido la longitud de alguno de los campos.
	\item \cdtIdRef{MSG7}{Registro repetido}: Se muestra en la pantalla \IUref{IU12.1.2}{Modificar Atributo} cuando se registre un atributo con un nombre que ya se encuentre registrado.
\end{Citemize}
