%--------------------------------------
\subsection{IU 6.1.1.1 Registrar trayectoria}

\subsubsection{Objetivo}
	Esta pantalla permite al actor registrar la información de una trayectoria, así como gestionar los pasos de la misma.
\subsubsection{Diseño}
	En la figura \IUref{IU6.1.1.1}{Registrar Trayectoria} se muestra la pantalla ''Registrar Trayectoria'' que permite registrar una trayectoria. El actor deberá ingresar la información solicitada, esto incluye los pasos de la trayectoria.
	Una vez ingresada la información solicitada, el actor deberá oprimir el botón \IUbutton{Aceptar} . El sistema validará y registrará la información solo si se han cumplido todas las reglas de negocio establecidas.
	
	Finalmente se mostrará el mensaje \cdtIdRef{MSG1}{Operación Exitosa} en la pantalla \IUref{IU6.1.1}{Gestionar Trayectorias}, para indicar que la información de la trayectoria se ha registrado correctamente.
	En la parte superior derecha, el sistema muestra el proyecto, el módulo y el caso de uso en el que actualmente se encuentra trabajando.

\IUfig[1]{interfaces/IU6-1-1-1registrarTray}{IU6.1.1.1}{Registrar Trayectoria}
\subsubsection{Comandos}
\begin{itemize}
	\item \IUbutton{Registrar}: Permite al actor solicitar el registro de un paso, dirige a la pantalla \IUref{IU6.1.1.1.1}{Registrar Paso}.
	\item \editar [Modificar]: Permite al actor solicitar la modificación de un paso, dirige a la pantalla \IUref{IU6.1.1.1.2}{Modificar Paso}.
	\item \eliminar [Eliminar]: Permite al actor solicitar la eliminación de un paso, dirige a una pantalla emergente.
	\item \IUbutton{Aceptar}: Permite al actor confirmar la modificación de la trayectoria, dirige a la pantalla \IUref{IU6.1.1}{Gestionar Trayectorias}.
	\item \IUbutton{Cancelar}: Permite al actor cancelar la modificación de la trayectoria, dirige a la pantalla \IUref{IU6.1.1}{Gestionar Trayectorias}.
\end{itemize}

\subsubsection{Mensajes}

\begin{Citemize}
	\item \cdtIdRef{MSG1}{Operación exitosa}: Se muestra en la pantalla \IUref{IU6.1.1}{Gestionar Trayectorias} para indicar que el registro fue exitoso.
	\item \cdtIdRef{MSG4}{Dato obligatorio}: Se muestra en la pantalla \IUref{IU6.1.1.1}{Registrar Trayectoria} cuando no se ha ingresado un dato marcado como obligatorio.
	\item \cdtIdRef{MSG5}{Dato incorrecto}: Se muestra en la pantalla \IUref{IU6.1.1.1}{Registrar Trayectoria} cuando el tipo de dato ingresado no cumple con el tipo de dato solicitado en el campo.
	\item \cdtIdRef{MSG6}{Longitud inválida}: Se muestra en la pantalla \IUref{IU6.1.1.1}{Registrar Trayectoria} cuando se ha excedido la longitud de alguno de los campos.
	\item \cdtIdRef{MSG7}{Registro repetido}: Se muestra en la pantalla \IUref{IU6.1.1.1}{Registrar Trayectoria} se haya ingresado una clave que ya esté registrada.
	\item \cdtIdRef{MSG14}{Dato no registrado}: Se muestra en la pantalla \IUref{IU6.1.1.1}{Registrar Trayectoria} cuando un elemento referenciado no existe en el sistema..
	\item \cdtIdRef{MSG18}{Caracteres inválidos}: Se muestra en la pantalla \IUref{IU6.1.1.1}{Registrar Trayectoria} cuando la clave de la trayectoria contiene un carácter no válido.
	\item \cdtIdRef{MSG19}{Token incorrecto}: Se muestra en la pantalla \IUref{IU6.1.1.1}{Registrar Trayectoria} cuando el token ingresado se encuentra estructurado de manera incorrecta.
\end{Citemize}
