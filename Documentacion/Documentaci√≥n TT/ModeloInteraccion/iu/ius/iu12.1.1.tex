%--------------------------------------
\section{IU 12.1.1 Registrar Atributo}

\subsection{Objetivo}
	Esta pantalla permite al actor registrar la información de un atributo.
\subsection{Diseño}
	En la figura \IUref{IU12.1.1}{Registrar Atributo} se muestra la pantalla ''Registrar Atributo'' que permite registrar un atributo perteneciente a un entidad en el sistema.
	Una vez ingresada la información solicitada, el actor deberá oprimir el botón \IUbutton{Aceptar} . El sistema validará y registrará la información solo si se han cumplido todas las reglas de negocio establecidas.
	
	Finalmente se mostrará el mensaje \cdtIdRef{MSG1}{Operación Exitosa} en la pantalla \IUref{IU12.1}{Registrar Entidad} o \IUref{IU12.2}{Modificar Entidad}, para indicar que la información del atributo se ha registrado correctamente.
	En la parte superior derecha, el sistema muestra el proyecto en el que se encuentra trabajando.
	
	\IUfig[1]{interfaces/IU12-1-1registrarAtributo}{IU12.1.1}{Registrar Atributo}
	
	Los datos solicitados dependerán del tipo de dato seleccionado:
	\begin{itemize}
		\item Tipo de dato ''Archivo'' el sistema solicitará los formatos válidos así como el tamaño máximo, como se muestra en la figura \IUref{IU12.1.1A}{Registrar Atributo: Archivo}.
		\IUfig[1]{interfaces/IU12-1-1AregistrarAtributo}{IU12.1.1A}{Registrar Atributo: Archivo}
		\item Tipo de dato ''Booleano'' o ''Fecha'', el sistema no solicitará ningún dato extra, como se muestra en la figura \IUref{IU12.1.1B}{Registrar Atributo: Booleano, Fecha}
		\IUfig[1]{interfaces/IU12-1-1BregistrarAtributo}{IU12.1.1B}{Registrar Atributo: Booleano, Fecha}
		\item Tipo de dato ''Cadena" o ''Entero'' o ''Flotante'', el sistema solicitará la longitud máxima, como se muestra en la figura \IUref{IU12.1.1C}{Registrar Atributo: Cadena, Entero, Flotante}
		\IUfig[1]{interfaces/IU12-1-1CregistrarAtributo}{IU12.1.1C}{Registrar Atributo: Cadena, Entero, Flotante}
		\item Tipo de dato ''Otro'', el sistema solicitará el tipo especificado, como se muestra en la figura \IUref{IU12.1.1D}{Registrar Atributo: Otro}
		\IUfig[1]{interfaces/IU12-1-1DregistrarAtributo}{IU12.1.1D}{Registrar Atributo: Otro}
	\end{itemize}

\subsection{Comandos}
\begin{itemize}
	\item \IUbutton{Aceptar}: Permite al actor guardar el registro del atributo, dirige a la pantalla \IUref{IU12.1}{Registrar Entidad} o \IUref{IU12.2}{Modificar Entidad}.
	\item \IUbutton{Cancelar}: Permite al actor cancelar el registro del atributo, dirige a la pantalla \IUref{IU12.1}{Registrar Entidad} o \IUref{IU12.2}{Modificar Entidad}.
\end{itemize}

\subsection{Mensajes}

\begin{Citemize}
	\item \cdtIdRef{MSG4}{Dato obligatorio}: Se muestra en la pantalla \IUref{IU12.1.1}{Registrar Atributo} cuando no se ha ingresado un dato marcado como obligatorio.
	\item \cdtIdRef{MSG5}{Dato incorrecto}: Se muestra en la pantalla \IUref{IU12.1.1}{Registrar Atributo} cuando el tipo de dato ingresado no cumple con el tipo de dato solicitado en el campo.
	\item \cdtIdRef{MSG6}{Longitud inválida}: Se muestra en la pantalla \IUref{IU12.1.1}{Registrar Atributo} cuando se ha excedido la longitud de alguno de los campos.
	\item \cdtIdRef{MSG7}{Registro repetido}: Se muestra en la pantalla \IUref{IU12.1.1}{Registrar Atributo} cuando se registre un atributo con un nombre que ya se encuentre registrado.
\end{Citemize}
