%--------------------------------------
\subsection{IU 6.1.2 Registrar Precondición}

\subsection{Objetivo}
	Esta pantalla permite al actor registrar una precondición.
\subsection{Diseño}
	En la figura \IUref{IU6.1.2}{Registrar Precondición} se muestra la pantalla ''Registrar Precondición'' que permite registrar una precondición. Una vez ingresada la redacción solicitada en la pantalla, el actor deberá oprimir el botón \IUbutton{Aceptar} . El sistema validará y agregará la precondición a la tabla de ''Precondiciones'' solo si se han cumplido todas las reglas de negocio establecidas.

\IUfig[1]{interfaces/IU6-1-2registrarPrecondicion}{IU6.1.2}{Registrar Precondición}
\subsection{Comandos}
\begin{itemize}
	\item \IUbutton{Aceptar}: Permite al actor guardar el registro de la precondición, dirige a la pantalla \IUref{IU6.1}{Registrar Caso de uso} o a la pantalla \IUref{IU6.2}{Modificar Caso de uso}, según corresponda.
	\item \IUbutton{Cancelar}: Permite al actor cancelar el registro de la precondición, dirige a la pantalla \IUref{IU6.1}{Registrar Caso de uso} o a la pantalla \IUref{IU6.2}{Modificar Caso de uso}, según corresponda
\end{itemize}

\subsection{Mensajes}

\begin{Citemize}
	\item \cdtIdRef{MSG4}{Dato obligatorio}: Se muestra en la pantalla \IUref{IU6.1.2}{Registrar precondición} cuando no se ha ingresado un dato marcado como obligatorio.
	\item \cdtIdRef{MSG6}{Longitud inválida}: Se muestra en la pantalla \IUref{IU6.1.2}{Registrar precondición} cuando se ha excedido la longitud de alguno de los campos.
\end{Citemize}
