	\begin{UseCase}{CU1}{Iniciar sesión}{
		El actor se ingresa con su nombre y contraseña a su perfil y hacer uso de las funciones que le competen.
	}
		\UCitem{Versión}{\color{Gray}0.1}
		\UCitem{Actor}{\hyperlink{jefe}{Líder de proyecto}, \hyperlink{analista}{Analista}, \hyperlink{admin}{Administrador}}
		\UCitem{Propósito}{Iniciar sesión en el sistema.}
		\UCitem{Entradas}{\begin{itemize}
				\item \cdtRef{colaboradorEntidad:correoColaborador}{Correo}: Se escribe desde el teclado
				\item \cdtRef{colaboradorEntidad:passColaborador}{Contraseña}: Se escribe desde el teclado
		\end{itemize}}
		\UCitem{Salidas}{Ninguna}
		\UCitem{Destino}{Pantalla}
		\UCitem{Precondiciones}{Que el actor se encuentre registrado en el sistema.}
		\UCitem{Postcondiciones}{El actor podrá hacer uso del sistema.}
		\UCitem{Errores}{\begin{itemize}
		\item \cdtIdRef{MSG4}{Dato Obligatorio}: Se muestra en la pantalla \IUref{IU1}{Iniciar Sesión} cuando no se ha ingresado un dato marcado como obligatorio.
		\item \cdtIdRef{MSG6}{Longitud inválida}: Se muestra en la pantalla \IUref{IU1}{Iniciar Sesión} cuando se ha excedido la longitud de alguno de los campos.
		\item \cdtIdRef{MSG31}{Correo electrónico y/o contraseña incorrectos}: Se muestra en la pantalla \IUref{IU1}{Iniciar Sesión} cuando el correo electrónico y/o la contraseña ingresada son incorrectos.
		\end{itemize}
		}
		\UCitem{Tipo}{Caso de uso primario}
	\end{UseCase}
%--------------------------------------
	\begin{UCtrayectoria}
		\UCpaso[\UCactor] Solicita ingresar al sistema a través de la URL.
		\UCpaso[\UCsist] Muestra la pantalla \IUref{IU1}{Iniciar Sesión}.
		\UCpaso[\UCactor] Ingresa los datos solicitados. \label{P3}
		\UCpaso[\UCactor] Oprime el botón \IUbutton{Aceptar}.
		\UCpaso[\UCsist] Verifica que no se haya omitido ningún campo marcado como obligatorio con base en la regla de negocio \BRref{RN8}{Datos Obligatorios}. \Trayref{A}
		\UCpaso[\UCsist] Verifica que los datos cumplan con el formato y el tipo de dato requerido, con base en la regla de negocio \BRref{RN7}{Información correcta}. \Trayref{B} \Trayref{C}
		\UCpaso[\UCsist] Verifica que el actor se encuentre registrado en el sistema. \Trayref{D}
		\UCpaso[\UCsist] Verifica que la contraseña ingresada corresponda con la del usuario. \Trayref{D}
		\UCpaso[\UCsist] Muestra la pantalla IU 4 Gestionar proyectos.	
	\end{UCtrayectoria}		
%--------------------------------------
		\begin{UCtrayectoriaA}{A}{El actor no ingresó un dato marcado como obligatorio}
			\UCpaso[\UCsist] Muestra el mensaje \cdtIdRef{MSG4}{Dato Obligatorio} y señala el campo que presenta el error en la pantalla \IUref{IU1}{Iniciar Sesión}.
			\UCpaso[\UCactor] Regresa al paso \ref{P3} de la Trayectoria Principal.
		\end{UCtrayectoriaA}

%--------------------------------------
		\begin{UCtrayectoriaA}{B}{El actor ingresó un dato con un número de caracteres fuera del rango permitido}
	\UCpaso[\UCsist] Muestra el \cdtIdRef{MSG6}{Longitud inválida} y señala el campo que presenta el error en la pantalla \IUref{IU1}{Iniciar Sesión}.
	\UCpaso[\UCactor] Regresa al paso \ref{P3} de la Trayectoria Principal.
\end{UCtrayectoriaA}

	\begin{UCtrayectoriaA}{C}{El actor ingresó un tipo de dato incorrecto.}
		\UCpaso[\UCsist] Muestra el mensaje \cdtIdRef{MSG29}{Formato incorrecto} y señala el campo que presenta el dato inválido en la pantalla \IUref{IU1}{Iniciar sesión}, para indicar que se ha ingresado un tipo de dato inválido.
		\UCpaso Regresa al paso \ref{P3} de la trayectoria principal.
	\end{UCtrayectoriaA}
%--------------------------------------
		\begin{UCtrayectoriaA}{D}{El actor ingresó un dato incorrecto}
	\UCpaso[\UCsist] Muestra el mensaje \cdtIdRef{MSG31}{Correo electrónico y/o contraseña incorrectos} en la pantalla \IUref{IU1}{Iniciar Sesión} notificando que los datos ingresados son incorrectos.
	\UCpaso[\UCactor] Regresa al paso \ref{P3} de la Trayectoria Principal.
\end{UCtrayectoriaA}