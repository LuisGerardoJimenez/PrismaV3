	\begin{UseCase}{CU7.2}{Modificar Entidad}{
		Este caso de uso permite al analista modificar la información de una entidad, así como gestionar sus atributos.
	}
		\UCitem{Versión}{\color{Gray}0.1}
		\UCitem{Actor}{\hyperlink{jefe}{Líder de Análisis}, \hyperlink{analista}{Analista}}
		\UCitem{Propósito}{Modificar la información de una entidad y gestionar sus atributos.}
		\UCitem{Entradas}{
		\begin{itemize}
			\item \cdtRef{entidadEntidad:nombreEntidad}{Nombre:} Se escribe desde el teclado.
			\item \cdtRef{entidadEntidad:descripcionEntidad}{Descripción:} Se escribe desde el teclado.
		\end{itemize}	
		}
		\UCitem{Salidas}{\begin{itemize}
				\item \cdtRef{entidadEntidad:nombreEntidad}{Nombre:} Lo obtiene el sistema.
				\item \cdtRef{entidadEntidad:descripcionEntidad}{Descripción:} Lo obtiene el sistema.
				\item \cdtRef{atributoEntidad}{Atributos:} Tabla que muestra \cdtRef{atributoEntidad:nombreATR}{Nombre}, \cdtRef{atributoEntidad:obligatorioATR}{Obligatorio (si o no)} y \hyperlink{tTipoDatoP}{Tipo de Dato} de todos los los registros de los atributos.
				\item \cdtRef{claveATR}{Clave:} Lo calcula el sistema mediante la regla de negocio \BRref{RN12}{Identifcador de elemento}.
				\item \cdtIdRef{MSG1}{Operación exitosa}: Se muestra en la pantalla \IUref{IU12}{Gestionar Entidades} para indicar que la modificación fue exitosa.
		\end{itemize}}
		\UCitem{Destino}{Pantalla}
		\UCitem{Precondiciones}{
			\begin{itemize}
			\item Que la entidad no se encuentre asociada a un caso de uso en estado ''Liberado''.
			\end{itemize}
		}
		\UCitem{Postcondiciones}{Ninguna}
		\UCitem{Errores}{\begin{itemize}
		\item \cdtIdRef{MSG4}{Dato obligatorio}: Se muestra en la pantalla \IUref{IU12.2}{Modificar Entidad} cuando no se ha ingresado un dato marcado como obligatorio.
		\item \cdtIdRef{MSG29}{Formato incorrecto}: Se muestra en la pantalla \IUref{IU12.2}{Modificar Entidad} cuando el tipo de dato ingresado no cumple con el tipo de dato solicitado en el campo.
		\item \cdtIdRef{MSG6}{Longitud inválida}: Se muestra en la pantalla \IUref{IU12.2}{Modificar Entidad} cuando se ha excedido la longitud de alguno de los campos.
		\item \cdtIdRef{MSG7}{Registro repetido}: Se muestra en la pantalla \IUref{IU12.2}{Modificar Entidad} cuando se registre una entidad con un nombre que ya se encuentra registrado en el sistema.
		\item \cdtIdRef{MSG16}{Registro necesario}: Se muestra en la pantalla \IUref{IU12.2}{Modificar Entidad} cuando el actor no ingrese ningún atributo.
		\item \cdtIdRef{MSG18}{Caracteres inválidos}: Se muestra en la pantalla \IUref{IU12.2}{Modificar Entidad} cuando el nombre de la entidad contiene un carácter no válido
		\end{itemize}
		}
		\UCitem{Tipo}{Secundario, extiende del caso de uso \UCref{CU7}{Gestionar Entidades}.}
	\end{UseCase}
%--------------------------------------
	\begin{UCtrayectoria}
		\UCpaso[\UCactor] Solicita registrar una entidad oprimiendo el botón \editar de algún registro existente en la pantalla \IUref{IU12}{Gestionar Entidades}.
		\UCpaso[\UCsist] Obtiene la información de la entidad.
		\UCpaso[\UCsist] Verifica que la entidad pueda modificarse con base en la regla de negocio \BRref{RN5}{Modificación de elementos asociados a casos de uso liberados}. \Trayref{MENT-G}
		\UCpaso[\UCsist] Muestra la pantalla \IUref{IU12.2}{Modificar Entidad} con la información encontrada.
		\UCpaso[\UCactor] Ingresa la información solicitada en la pantalla. \label{CU7.2-P5}
		\UCpaso[\UCactor] Gestiona los atributos a través de los botones: \IUbutton{Registrar}, \editar y \eliminar. \label{CU7.2-P6}
		\UCpaso[\UCactor] Solicita guardar la información de la entidad oprimiendo el botón \IUbutton{Aceptar} de la pantalla \IUref{IU12.2}{Modificar Entidad}. \Trayref{MENT-A}
		\UCpaso[\UCsist] Verifica que el actor ingrese todos los campos obligatorios con base en la regla de negocio \BRref{RN8}{Datos obligatorios}. \Trayref{MENT-B}
		\UCpaso[\UCsist] Verifica que los datos requeridos sean proporcionados correctamente con base en la regla de negocio \BRref{RN7}{Información correcta}. \Trayref{MENT-C} \Trayref{MENT-D}
		\UCpaso[\UCsist] Verifica que el nombre de la entidad no se encuentre registrado en el sistema con base en la regla de negocio \BRref{RN6}{Unicidad de nombres}. \Trayref{MENT-E} 
		\UCpaso[\UCsist] Verifica que el actor haya ingresado al menos un atributo. \Trayref{MENT-F}
		\UCpaso[\UCsist] Modifica la información de la entidad en el sistema.
		\UCpaso[\UCsist] Muestra el mensaje \cdtIdRef{MSG1}{Operación exitosa} en la pantalla \IUref{IU12}{Gestionar Entidades} para indicar al actor que la modificación se ha realizado exitosamente.
	\end{UCtrayectoria}		
%--------------------------------------
	
	\begin{UCtrayectoriaA}{MENT-A}{El actor desea cancelar la operación.}
		\UCpaso[\UCactor] Solicita cancelar la operación oprimiendo el botón \IUbutton{Cancelar} de la pantalla \IUref{IU12.2}{Modificar Entidad}
		\UCpaso[\UCsist] Muestra la pantalla \IUref{IU12}{Gestionar Entidades}.
	\end{UCtrayectoriaA}

	\begin{UCtrayectoriaA}{MENT-B}{El actor no ingresó algún dato marcado como obligatorio.}
		\UCpaso[\UCsist] Muestra el mensaje \cdtIdRef{MSG4}{Dato obligatorio} y señala el campo que presenta el error en la pantalla \IUref{IU12.2}{Modificar Entidad}, indicando al actor que el dato es obligatorio.
		\UCpaso Regresa al paso \ref{CU7.2-P5} de la trayectoria principal.
	\end{UCtrayectoriaA}

	\begin{UCtrayectoriaA}{MENT-C}{El actor proporciona un dato que excede la longitud máxima.}
		\UCpaso[\UCsist] Muestra el mensaje \cdtIdRef{MSG6}{Longitud inválida} y señala el campo que excede la longitud en la pantalla \IUref{IU12.2}{Modificar Entidad}, para indicar que el dato excede el tamaño máximo permitido.
		\UCpaso Regresa al paso \ref{CU7.2-P5} de la trayectoria principal.
	\end{UCtrayectoriaA}

	\begin{UCtrayectoriaA}{MENT-D}{El actor ingresó un tipo de dato incorrecto.}
		\UCpaso[\UCsist] Muestra el mensaje \cdtIdRef{MSG29}{Formato incorrecto} y señala el campo que presenta el dato inválido en la pantalla \IUref{IU12.2}{Modificar entidad}, para indicar que se ha ingresado un tipo de dato inválido.
		\UCpaso Regresa al paso \ref{CU7.2-P5} de la trayectoria principal.
	\end{UCtrayectoriaA}
	
	\begin{UCtrayectoriaA}{MENT-E}{El actor ingresó un nombre de una entidad repetido.}
		\UCpaso[\UCsist] Muestra el mensaje \cdtIdRef{MSG7}{Registro repetido} y señala el campo que presenta la duplicidad en la pantalla \IUref{IU12.2}{Modificar Entidad}, indicando al actor que existe una entidad con el mismo nombre.
		\UCpaso Regresa al paso \ref{CU7.2-P5} de la trayectoria principal.
	\end{UCtrayectoriaA}

	\begin{UCtrayectoriaA}{MENT-F}{El actor no registró ningún atributo.}
	\UCpaso[\UCsist] Muestra el mensaje \cdtIdRef{MSG16}{Dato incorrecto} en la sección de atributos de la pantalla \IUref{IU12.2}{Modificar Entidad}.
	\UCpaso Regresa al paso \ref{CU7.2-P5} de la trayectoria principal.
	\end{UCtrayectoriaA}


	\begin{UCtrayectoriaA}{MENT-G}{La entidad no puede modificarse debido a que se encuentra asociado a casos de uso liberados.}
		\UCpaso[\UCsist] Oculta el botón \editar de la entidad que esta asociada a casos de uso liberados.
	\end{UCtrayectoriaA}

\subsubsection{Puntos de extensión}

\UCExtenssionPoint{El actor requiere registrar un atributo.}{Paso \ref{CU7.2-P6} de la trayectoria principal.}{\UCref{CU7.1.1}{Registrar Atributo}}
\UCExtenssionPoint{El actor requiere modificar un atributo.}{Paso \ref{CU7.2-P6} de la trayectoria principal.}{\UCref{CU7.1.2}{Modificar Atributo}}
\UCExtenssionPoint{El actor requiere eliminar un atributo.}{Paso \ref{CU7.2-P6} de la trayectoria principal.}{\UCref{CU7.1.3}{Eliminar Atributo}}