	\begin{UseCase}{CU12.1.1.1.1}{Modificar Paso}{
		Los pasos que describen una trayectoria deben indicar las acciones que realiza el sistema o el actor. Este caso de uso permite al analista modificar un paso de alguna trayectoria.
	}
		\UCitem{Versión}{\color{Gray}0.1}
		\UCitem{Actor}{\hyperlink{jefe}{Líder de Análisis}, \hyperlink{analista}{Analista}}
		\UCitem{Propósito}{Modificar los pasos de la trayectoria principal o de alguna trayectoria alternativa.}
		\UCitem{Entradas}{
		\begin{itemize}
			\item \cdtRef{entidadPaso:realizaPaso}{Quien realiza el paso}: Se escribe desde el teclado.
			\item Verbo: Se selecciona de una lista.
			\item \cdtRef{entidadPaso:redaccionPaso}{Redacción del paso:} Se escribe desde el teclado.
		\end{itemize}	
		}
		\UCitem{Salidas}{
			\begin{itemize}
				\item \cdtRef{entidadPaso:realizaPaso}{Quien realiza el paso}: Lo obtiene el sistema.
				\item Verbo: Lo obtiene el sistema.
				\item \cdtRef{entidadPaso:redaccionPaso}{Redacción del paso:} Lo obtiene el sistema.
			\end{itemize}
		}
		\UCitem{Destino}{Pantalla}
		\UCitem{Precondiciones}{Ninguna}
		\UCitem{Postcondiciones}{Ninguna}
		\UCitem{Errores}{\begin{itemize}
		\item \cdtIdRef{MSG4}{Dato obligatorio}: Se muestra en la pantalla \IUref{IU6.1.1.1.1}{Registrar Paso} cuando no se ha ingresado un dato marcado como obligatorio.
		\item \cdtIdRef{MSG6}{Longitud inválida}: Se muestra en la pantalla \IUref{IU6.1.1.1.1}{Registrar Paso} cuando se ha excedido la longitud de alguno de los campos.
		\end{itemize}.}
		\UCitem{Tipo}{Secundario, extiende del caso de uso \UCref{CU12.1.1.1}{Registrar Trayectoria} y \UCref{CU12.1.1.2}{Modificar Trayectoria}.}
	\end{UseCase}
%--------------------------------------
	\begin{UCtrayectoria}
		\UCpaso[\UCactor] Solicita modificar un paso oprimiendo el botón \editar de la pantalla \IUref{IU6.1.1.1}{Registrar Trayectoria} o \IUref{IU6.1.1.2}{Modificar Trayectoria}.
		\UCpaso[\UCsist] Muestra la pantalla \IUref{IU6.1.1.1.2}{Modificar Paso}.
		\UCpaso[\UCactor] Modifica quien realiza el paso. \label{CU12.1.1.1.2-P3}
		\UCpaso[\UCactor] Modifica el verbo.
		\UCpaso[\UCactor] Modifica la redacción del paso. \Trayref{MPA-A} \Trayref{MPA-B} \Trayref{MPA-C} \Trayref{MPA-D} \Trayref{MPA-E} \Trayref{MPA-F} \Trayref{MPA-G} \Trayref{MPA-H} \Trayref{MPA-I} \Trayref{MPA-J} \Trayref{MPA-K} \label{CU12.1.1.1.2-P5}
		\UCpaso[\UCactor] Solicita modificar el paso oprimiendo el botón \IUbutton{Aceptar} de la pantalla \IUref{IU6.1.1.1.2}{Modificar Paso}. \Trayref{MPA-L} 
		\UCpaso[\UCsist] Verifica que el actor ingrese todos los campos obligatorios con base en la regla de negocio \BRref{RN8}{Datos obligatorios}. \Trayref{MPA-M}
		\UCpaso[\UCsist] Verifica que los datos requeridos sean proporcionados correctamente con base en la regla de negocio \BRref{RN7}{Información correcta}. \Trayref{MPA-N}
		\UCpaso[\UCsist] Modifica el paso en la tabla de la pantalla \IUref{IU6.1.1.1}{Registrar Trayectoria} o \IUref{IU6.1.1.2}{Modificar Trayectoria}
	\end{UCtrayectoria}		
%--------------------------------------
	
	\begin{UCtrayectoriaA}{MPA-A}{El actor desea seleccionar un actor.}
		\UCpaso[\UCactor] Ingresa el token {\em ACT·}.
		\UCpaso[\UCsist] Obtiene los actores registrados en el proyecto. 
		\UCpaso[\UCsist] Muestra una lista con los actores encontrados.
		\UCpaso[\UCactor] Selecciona un actor de la lista.
		\UCpaso[\UCsist] Verifica que el nombre del actor seleccionado no contenga espacios. \Trayref{MPA-Ñ}
		\UCpaso[\UCsist] Agrega el nombre del actor al texto.
		\UCpaso Continúa en el paso \ref{CU12.1.1.1.2-P5} de la trayectoria principal.
	\end{UCtrayectoriaA}

	\begin{UCtrayectoriaA}{MPA-B}{El actor desea seleccionar un término del glosario.}
		\UCpaso[\UCactor] Ingresa el token {\em GLS·}.
		\UCpaso[\UCsist] Obtiene los términos del glosario registradas en el proyecto. 
		\UCpaso[\UCsist] Muestra una lista con los términos del glosario encontrados.
		\UCpaso[\UCactor] Selecciona un término del glosario de la lista.
		\UCpaso[\UCsist] Verifica que el nombre del término seleccionado no contenga espacios. \Trayref{MPA-Ñ}
		\UCpaso[\UCsist] Agrega el nombre del término del glosario al texto.
		\UCpaso Continúa en el paso \ref{CU12.1.1.1.2-P5}.
	\end{UCtrayectoriaA}

	\begin{UCtrayectoriaA}{MPA-C}{El actor desea seleccionar Atributo.}
		\UCpaso[\UCactor] Ingresa el token {\em ATR·}. 
		\UCpaso[\UCsist] Obtiene los atributos de las entidades registradas en el proyecto.
		\UCpaso[\UCsist] Muestra una lista con los atributos encontrados.
		\UCpaso[\UCactor] Selecciona un atributo de la lista.
		\UCpaso[\UCsist] Verifica que el nombre de la entidad a la que pertenece el atributo no contenga espacios. \Trayref{MPA-Ñ}
		\UCpaso[\UCsist] Verifica que el nombre del atributo no contenga espacios. \Trayref{MPA-Ñ}
		\UCpaso[\UCsist] Agrega el nombre de la entidad a la que pertenece el atributo al texto, seguido del signo '':''.
		\UCpaso[\UCsist] Agrega el nombre del atributo al texto.
		\UCpaso Continúa en el paso \ref{CU12.1.1.1.2-P5} de la trayectoria principal, según corresponda.
	\end{UCtrayectoriaA}

	\begin{UCtrayectoriaA}{MPA-D}{El actor desea seleccionar un mensaje.}
		\UCpaso[\UCactor] Ingresa el token {\em MSG·}. 
		\UCpaso[\UCsist] Obtiene los mensaje registrados en el proyecto.
		\UCpaso[\UCsist] Muestra una lista con los mensajes encontrados.
		\UCpaso[\UCactor] Selecciona un mensaje de la lista.
		\UCpaso[\UCsist] Verifica que el nombre del mensaje seleccionado no contenga espacios. \Trayref{MPA-Ñ}
		\UCpaso[\UCsist] Agrega el número del mensaje al texto, seguido del signo '':''.
		\UCpaso[\UCsist] Agrega el nombre del mensaje al texto.
		\UCpaso Continúa en el paso \ref{CU12.1.1.1.2-P5} de la trayectoria principal.
	\end{UCtrayectoriaA}

	\begin{UCtrayectoriaA}{MPA-E}{El actor desea seleccionar una regla de negocio.}
		\UCpaso[\UCactor] Ingresa el token {\em RN·}. 
		\UCpaso[\UCsist] Obtiene las reglas de negocio registradas en el proyecto.
		\UCpaso[\UCsist] Muestra una lista con las reglas de negocio encontradas.
		\UCpaso[\UCactor] Selecciona una regla de negocio de la lista.
		\UCpaso[\UCsist] Verifica que el nombre de la regla de negocio seleccionada no contenga espacios. \Trayref{MPA-Ñ}
		\UCpaso[\UCsist] Agrega el número de la regla de negocio al texto, seguido del signo '':''.
		\UCpaso[\UCsist] Agrega el nombre de la regla de negocio al texto.
		\UCpaso Continúa en el paso \ref{CU12.1.1.1.2-P5} de la trayectoria principal.
	\end{UCtrayectoriaA}

	\begin{UCtrayectoriaA}{MPA-F}{El actor desea seleccionar una entidad.}
		\UCpaso[\UCactor] Ingresa el token {\em ENT·}. 
		\UCpaso[\UCsist] Obtiene las entidades registradas en el proyecto.
		\UCpaso[\UCsist] Muestra una lista con las entidades encontradas.
		\UCpaso[\UCactor] Selecciona una entidad de la lista.
		\UCpaso[\UCsist] Verifica que el nombre de la entidad seleccionada no contenga espacios. \Trayref{MPA-Ñ}
		\UCpaso[\UCsist] Agrega el nombre de la entidad al texto.
		\UCpaso Continúa en el paso \ref{CU12.1.1.1.2-P5} de la trayectoria principal.
	\end{UCtrayectoriaA}

	\begin{UCtrayectoriaA}{MPA-G}{El actor desea seleccionar un caso de uso.}
		\UCpaso[\UCactor] Ingresa el token {\em CU·}. 
		\UCpaso[\UCsist] Obtiene los casos de uso registrados en el proyecto.
		\UCpaso[\UCsist] Muestra una lista con los casos de uso encontrados.
		\UCpaso[\UCactor] Selecciona un caso de uso de la lista.
		\UCpaso[\UCsist] Verifica que el nombre del caso de uso seleccionado no contenga espacios. \Trayref{MPA-Ñ}
		\UCpaso[\UCsist] Agrega el número del caso de uso al texto, seguido del signo '':''.
		\UCpaso[\UCsist] Agrega el nombre del caso de uso al texto.
		\UCpaso Continúa en el paso \ref{CU12.1.1.1.2-P5} de la trayectoria principal.
	\end{UCtrayectoriaA}

	\begin{UCtrayectoriaA}{MPA-H}{El actor desea seleccionar una pantalla.}
		\UCpaso[\UCactor] Ingresa el token {\em CU·}. 
		\UCpaso[\UCsist] Obtiene las pantallas registradas del proyecto.
		\UCpaso[\UCsist] Muestra una lista con las pantallas encontradas.
		\UCpaso[\UCactor] Selecciona una pantalla de la lista.
		\UCpaso[\UCsist] Verifica que el nombre de la pantalla seleccionada no contenga espacios. \Trayref{MPA-Ñ}
		\UCpaso[\UCsist] Agrega el número de la pantalla al texto, seguido del signo '':''.
		\UCpaso[\UCsist] Agrega el nombre de la pantalla al texto.
		\UCpaso Continúa en el paso \ref{CU12.1.1.1.2-P5} de la trayectoria principal.
	\end{UCtrayectoriaA}

	\begin{UCtrayectoriaA}{MPA-I}{El actor desea seleccionar un paso.}
		\UCpaso[\UCactor] Ingresa el token {\em P·}. 
		\UCpaso[\UCsist] Obtiene los pasos del caso de uso.
		\UCpaso[\UCsist] Muestra una lista con los pasos encontradas.
		\UCpaso[\UCactor] Selecciona un paso de la lista.
		\UCpaso[\UCsist] Verifica que el nombre del caso de uso al que pertenece el paso no contenga espacios. \Trayref{MPA-Ñ}
		\UCpaso[\UCsist] Agrega la clave del caso de uso al que pertenece el paso al texto, seguido del signo ''·''.
		\UCpaso[\UCsist] Agrega el número del caso de uso al texto, seguido del signo '':''.
		\UCpaso[\UCsist] Agrega el nombre del caso de uso al texto, seguido del signo '':''.
		\UCpaso[\UCsist] Agrega la clave de la trayectoria a la que pertenece el paso al texto, seguido del signo ''·''.
		\UCpaso[\UCsist] Agrega el número del paso seleccionado al texto.
		\UCpaso Continúa en el paso \ref{CU12.1.1.1.2-P5} de la trayectoria principal.
	\end{UCtrayectoriaA}

	\begin{UCtrayectoriaA}{MPA-J}{El actor desea seleccionar una trayectoria.}
		\UCpaso[\UCactor] Ingresa el token {\em TRAY·}. 
		\UCpaso[\UCsist] Obtiene las trayectorias del caso de uso.
		\UCpaso[\UCsist] Muestra una lista con las trayectorias encontradas.
		\UCpaso[\UCactor] Selecciona una trayectoria de la lista.
		\UCpaso[\UCsist] Verifica que el nombre del caso de uso al que pertenece la trayectoria no contenga espacios. \Trayref{MPA-Ñ}
		\UCpaso[\UCsist] Agrega la clave del caso de uso al que pertenece la trayectoria al texto, seguido del signo ''·''.
		\UCpaso[\UCsist] Agrega el número del caso de uso al texto, seguido del signo '':''.
		\UCpaso[\UCsist] Agrega el nombre del caso de uso al texto, seguido del signo '':''.
		\UCpaso[\UCsist] Agrega la clave de la trayectoria seleccionada.
		\UCpaso Continúa en el paso \ref{CU12.1.1.1.2-P5} de la trayectoria principal.
	\end{UCtrayectoriaA}

	\begin{UCtrayectoriaA}{MPA-K}{El actor desea seleccionar una acción.}
		\UCpaso[\UCactor] Ingresa el token {\em ACC·}. 
		\UCpaso[\UCsist] Obtiene las acciones de las pantallas del proyecto.
		\UCpaso[\UCsist] Muestra una lista de las acciones encontradas.
		\UCpaso[\UCactor] Selecciona una acción de la lista.
		\UCpaso[\UCsist] Verifica que el nombre de la pantalla a la que pertenece la acción no contenga espacios. \Trayref{MPA-Ñ}
		\UCpaso[\UCsist] Agrega la clave de la pantalla a la que pertenece la acción al texto, seguido del signo ''·''.
		\UCpaso[\UCsist] Agrega el número de la pantalla al texto, seguido del signo '':''.
		\UCpaso[\UCsist] Agrega el nombre de la pantalla al texto, seguido del signo '':''.
		\UCpaso[\UCsist] Agrega el nombre de la acción seleccionada al texto.
		\UCpaso Continúa en el paso \ref{CU12.1.1.1.2-P5} de la trayectoria principal.
	\end{UCtrayectoriaA}

	\begin{UCtrayectoriaA}{MPA-L}{El actor desea cancelar la operación.}
		\UCpaso[\UCactor] Solicita cancelar la operación oprimiendo el botón \IUbutton{Cancelar} de la pantalla \IUref{IU6.1.1.1.2}{Modificar Paso}.
		\UCpaso[\UCsist] Muestra la pantalla \IUref{IU6.1.1.1}{Registrar Trayectoria} o \IUref{IU6.1.1.2}{Modificar Trayectoria}.
	\end{UCtrayectoriaA}

	\begin{UCtrayectoriaA}{MPA-M}{El actor no ingresó algún dato marcado como obligatorio.}
		\UCpaso[\UCsist] Muestra el mensaje \cdtIdRef{MSG4}{Dato obligatorio} y señala el campo que presenta el error en la pantalla \IUref{IU6.1.1.1.2}{Modificar Paso}, indicando al actor que el dato es obligatorio.
		\UCpaso Regresa al paso \ref{CU12.1.1.1.2-P3} de la trayectoria principal.
	\end{UCtrayectoriaA}

	\begin{UCtrayectoriaA}{MPA-N}{El actor proporciona un dato que excede la longitud máxima.}
		\UCpaso[\UCsist] Muestra el mensaje \cdtIdRef{MSG6}{Longitud inválida} y señala el campo que excede la longitud en la pantalla \IUref{IU6.1.1.1.2}{Modificar Paso}, para indicar que el dato excede el tamaño máximo permitido.
		\UCpaso Regresa al paso \ref{CU12.1.1.1.2-P3} de la trayectoria principal.
	\end{UCtrayectoriaA}

	\begin{UCtrayectoriaA}{MPA-Ñ}{El texto contiene espacios.}
		\UCpaso[\UCsist] Sustituye los espacios por guiones bajos.
	\end{UCtrayectoriaA}