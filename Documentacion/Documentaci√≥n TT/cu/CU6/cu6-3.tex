	\begin{UseCase}{CU6.3}{Eliminar Término}{
		Este caso de uso permite al actor eliminar del sistema un término del glosario.
	}
		\UCitem{Versión}{\color{Gray}0.1}
		\UCitem{Actor}{\hyperlink{jefe}{Líder de Análisis}, \hyperlink{analista}{Analista}}
		\UCitem{Propósito}{Eliminar la información de un término del glosario.}
		\UCitem{Entradas}{Ninguna}
		\UCitem{Salidas}{\begin{itemize}
				\item \cdtIdRef{MSG1}{Operación exitosa}: Se muestra en la pantalla \IUref{IU11}{Gestionar Términos de glosario} para indicar que el término fue eliminado correctamente.
				\item \cdtIdRef{MSG10}{Confirmar eliminación}: Se muestra en la pantalla \IUref{IU11}{Gestionar Términos de glosario} para que el actor confirme la eliminación.
		\end{itemize}}
		\UCitem{Destino}{Pantalla}
		\UCitem{Precondiciones}{Ninguna}
		\UCitem{Postcondiciones}{
		\begin{itemize}
			\item Se eliminará el término del glosario del sistema.
		\end{itemize}
		}
		\UCitem{Errores}{\begin{itemize}
		\item \cdtIdRef{MSG12}{Ha ocurrido un error}: Se muestra en una pantalla emergente cuando no se pueda eliminar el término del glosario debido a que está siendo referenciado en algún caso de uso.
		\item \cdtIdRef{MSG13}{Eliminación no permitida}: Se muestra en la pantalla \IUref{IU11}{Gestionar Términos del glosario} cuando el término del glosario no se encuentre en un estado que permita la eliminación.
		\end{itemize}
		}
		\UCitem{Tipo}{Secundario, extiende del caso de uso \UCref{CU6}{Gestionar Términos del glosario}.}
	\end{UseCase}
%--------------------------------------
	\begin{UCtrayectoria}
		\UCpaso[\UCactor] Solicita eliminar un término del glosario oprimiendo el botón \eliminar del registro que desea eliminar de la pantalla \IUref{IU11}{Gestionar Términos de glosario}.
		\UCpaso[\UCsist] Verifica que el término del glosario pueda eliminarse de acuerdo a la regla de negocio \BRref{RN18}{Eliminación de elementos}. \Trayref{ET-A}
		\UCpaso[\UCsist] Verifica que ningún caso de uso se encuentre asociado al término del glosario. \Trayref{ET-B}
		\UCpaso[\UCsist] Muestra el mensaje emergente \cdtIdRef{MSG10}{Confirmar eliminación} con los botones \IUbutton{Aceptar} y \IUbutton{Cancelar} en la pantalla \IUref{IU11}{Gestionar Términos del glosario}.
		\UCpaso[\UCactor] Confirma la eliminación del término oprimiendo el botón \IUbutton{Aceptar}. \Trayref{ET-C}
		\UCpaso[\UCsist] Elimina la información referente al término del glosario.
		\UCpaso[\UCsist] Muestra el mensaje \cdtIdRef{MSG1}{Operación exitosa} en la pantalla \IUref{IU11}{Gestionar Términos del glosario} para indicar al actor que el registro se ha eliminado exitosamente.
	\end{UCtrayectoria}		
%--------------------------------------
	
	\begin{UCtrayectoriaA}{ET-A}{El término esta asociado un caso de uso liberado.}
		\UCpaso[\UCsist] Oculta el botón \eliminar de la entidad que esta asociada a casos de uso liberados.
	\end{UCtrayectoriaA}

	\begin{UCtrayectoriaA}{ET-B}{El término del glosario está siendo referenciado en un caso de uso.}
		\UCpaso[\UCsist] Muestra el mensaje \cdtIdRef{MSG13}{Eliminación no permitida} en la pantalla \IUref{IU11}{Gestionar Términos del glosario} en una pantalla emergente con la lista de casos de uso que están referenciando al término del glosario.
		\UCpaso[\UCactor] Oprime el botón \IUbutton{Aceptar} de la pantalla emergente.
		\UCpaso[\UCsist] Muestra la pantalla \IUref{IU11}{Gestionar Términos del glosario}.
	\end{UCtrayectoriaA}


	\begin{UCtrayectoriaA}{ET-C}{El actor desea cancelar la operación.}
		\UCpaso[\UCactor] Oprime el botón \IUbutton{Cancelar} de la pantalla emergente.
		\UCpaso[\UCsist] Muestra la pantalla \IUref{IU11}{Gestionar Términos del glosario}.
	\end{UCtrayectoriaA}
	

