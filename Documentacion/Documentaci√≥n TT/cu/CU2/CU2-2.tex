	\begin{UseCase}{CU2.2}{Modificar proyecto}{
		Este caso de uso permite al actor modificar la información de un proyecto en el sistema.
	}
		\UCitem{Versión}{\color{Gray}0.1}
		\UCitem{Actor}{\hyperlink{admin}{Administrador}}
		\UCitem{Propósito}{Modificar la información de un proyecto.}
		\UCitem{Entradas}{
		\begin{itemize}
			\item \cdtRef{proyectoEntidad:claveProyecto}{Clave:} Se escribe desde el teclado.
			\item \cdtRef{proyectoEntidad:nombreProyecto}{Nombre:} Se escribe desde el teclado.
			\item \cdtRef{proyectoEntidad:fechaIProyecto}{Fecha de inicio:} Se selecciona de un calendario.
			\item \cdtRef{proyectoEntidad:fechaFinProyecto}{Fecha de término:} Se selecciona de un calendario.
			\item \cdtRef{proyectoEntidad:fechaIPProyecto}{Fecha de inicio programada:} Se selecciona de un calendario.
			\item \cdtRef{proyectoEntidad:fechaFinPProyecto}{Fecha de término programada:} Se selecciona de un calendario.
			\item \cdtRef{proyectoEntidad:liderProyecto}{Líder del Proyecto:} Se selecciona de una lista.
			\item \cdtRef{proyectoEntidad:descripcionProyecto}{Descripción:} Se escribe desde el teclado.
			\item \cdtRef{proyectoEntidad:contraparteProyecto}{Contraparte} Se escribe desde el teclado.
			\item \cdtRef{proyectoEntidad:presupuestoProyecto}{Presupuesto:} Se escribe desde el teclado.
			\item \hyperlink{tEdoProy}{Estado del Proyecto:} Se selecciona de un lista.
		\end{itemize}	
		}
		\UCitem{Salidas}{\begin{itemize}
				\item \cdtRef{proyectoEntidad:claveProyecto}{Clave:} Lo obtiene el sistema.
				\item \cdtRef{proyectoEntidad:nombreProyecto}{Nombre:} Lo obtiene el sistema.
				\item \cdtRef{proyectoEntidad:fechaIProyecto}{Fecha de inicio:} Lo obtiene el sistema.
				\item \cdtRef{proyectoEntidad:fechaFinProyecto}{Fecha de término:} Lo obtiene el sistema.
				\item \cdtRef{proyectoEntidad:fechaIPProyecto}{Fecha de inicio programada:} Lo obtiene el sistema.
				\item \cdtRef{proyectoEntidad:fechaFinPProyecto}{Fecha de término programada:} Lo obtiene el sistema.
				\item \cdtRef{proyectoEntidad:liderProyecto}{Líder del Proyecto:} Lo obtiene el sistema.
				\item \cdtRef{proyectoEntidad:descripcionProyecto}{Descripción:} Lo obtiene el sistema.
				\item \cdtRef{proyectoEntidad:contraparteProyecto}{Contraparte} Lo obtiene el sistema.
				\item \cdtRef{proyectoEntidad:presupuestoProyecto}{Presupuesto:} Lo obtiene el sistema.
				\item \hyperlink{tEdoProy}{Estado del Proyecto:} Lo obtiene el sistema.
				\item \cdtIdRef{MSG1}{Operación exitosa}: Se muestra en la pantalla \IUref{IU2}{Gestionar proyectos de Administrador} para indicar que el registro fue exitoso.
		\end{itemize}}
		\UCitem{Destino}{Pantalla}
		\UCitem{Precondiciones}{
		\begin{itemize}
			\item Que el proyecto se encuentre en estado ''En negociación'' o ''Iniciado''.
		\end{itemize}
		}
		\UCitem{Postcondiciones}{
		\begin{itemize}
			\item Se modificará la información del proyecto en el sistema.
		\end{itemize}
		}
		\UCitem{Errores}{\begin{itemize}
		\item \cdtIdRef{MSG4}{Dato obligatorio}: Se muestra en la pantalla \IUref{IU2.2}{Modificar proyecto} cuando no se ha ingresado un dato marcado como obligatorio.
		\item \cdtIdRef{MSG29}{Formato incorrecto}: Se muestra en la pantalla \IUref{IU2.2}{Modificar proyecto} cuando el tipo de dato ingresado no cumple con el tipo de dato solicitado en
		el campo.
		\item \cdtIdRef{MSG6}{Longitud inválida}: Se muestra en la pantalla \IUref{IU2.2}{Modificar proyecto} cuando se ha excedido la longitud de alguno de los campos.
		\item \cdtIdRef{MSG7}{Registro repetido}: Se muestra en la pantalla \IUref{IU2.2}{Modificar proyecto} cuando se registre un proyecto con un nombre o clave que ya exista.
		\item \cdtIdRef{MSG12}{Ha ocurrido un error}: Se muestra en la pantalla \IUref{IU2}{Gestionar proyectos de Administrador} cuando no exista información de los estados de un proyecto.
		\item \cdtIdRef{MSG17}{Falta información}: Se muestra en la pantalla \IUref{IU2}{Gestionar proyectos de Administrador} cuando no existan colaboradores registrados.
		\item \cdtIdRef{MSG26}{Orden de fechas}: Se muestra en la pantalla \IUref{IU2.1}{Registrar proyecto} cuando el actor ingrese fechas de término que no son posteriores a las fechas
		de inicio correspondientes.
		\end{itemize}
		}
		\UCitem{Tipo}{Secundario, extiende del caso de uso \UCref{CU2}{Gestionar proyectos de Administrador}}
	\end{UseCase}
%--------------------------------------
	\begin{UCtrayectoria}
		\UCpaso[\UCactor] Solicita registrar un proyecto oprimiendo el botón \editar de la pantalla \IUref{IU2}{Gestionar proyectos de Administrador}.
		\UCpaso[\UCsist] Verifica que exista información referente a los estados de un proyecto, con base en la regla de negocio \BRref{RN20}{Verificación de catálogos}. \hyperlink{CU2-2:TAA}{[Trayectoria A]}
		\UCpaso[\UCsist] Verifica que exista al menos un colaborador, con base en la regla de negocio \BRref{RN20}{Verificación de catálogos}. \hyperlink{CU2-2:TAB}{[Trayectoria B]}
		\UCpaso[\UCsist] Obtiene la información del proyecto seleccionado.
		\UCpaso[\UCsist] Muestra la pantalla \IUref{IU2.2}{Modificar proyecto}.
		\UCpaso[\UCactor] Ingresa la información solicitada en la pantalla. \label{CU2.2-P5}
		\UCpaso[\UCactor] Solicita guardar el proyecto oprimiendo el botón \IUbutton{Aceptar} de la pantalla \IUref{IU2.2}{Modificar proyecto}. \hyperlink{CU2-2:TAC}{[Trayectoria C]}
		\UCpaso[\UCsist] Verifica que el actor ingrese todos los campos obligatorios con base en la regla de negocio \BRref{RN8}{Datos obligatorios}. \hyperlink{CU2-2:TAD}{[Trayectoria D]}
		\UCpaso[\UCsist] Verificar que los datos ingresados cumpla con la longitud correcta, con base en la regla de negocio \BRref{RN37}{Longitud de datos}. \hyperlink{CU2-2:TAE}{[Trayectoria E]}
		\UCpaso[\UCsist] Verifica que los datos ingresados cumplan con el formato y el tipo de dato requerido, con base en la regla de negocio \BRref{RN7}{Información correcta}. \hyperlink{CU2-2:TAF}{[Trayectoria F]}
		\UCpaso[\UCsist] Verifica que la clave del proyecto no se encuentre registrada en el sistema con base en la regla de negocio \BRref{RN22}{Unicidad de la clave del Proyecto}. \hyperlink{CU2-2:TAG}{[Trayectoria G]}
		\UCpaso[\UCsist] Verifica que el nombre del proyecto no se encuentre registrado en el sistema con base en la regla de negocio \BRref{RN6}{Unicidad de nombres}. \hyperlink{CU2-2:TAH}{[Trayectoria H]}
		\UCpaso[\UCsist] Verifica que la fecha de término programada sea posterior a la fecha de inicio programada. \hyperlink{CU2-2:TAI}{[Trayectoria I]}
		\UCpaso[\UCsist] Actualiza la información del proyecto en el sistema
		\UCpaso[\UCsist] Muestra el mensaje \cdtIdRef{MSG1}{Operación exitosa} en la pantalla \IUref{IU2}{Gestionar proyectos de Administrador} para indicar al actor que la modificación se ha realizado exitosamente.
	\end{UCtrayectoria}		
%--------------------------------------
	\hypertarget{CU2-2:TAA}{\textbf{Trayectoria alternativa A}}\\
	\noindent \textbf{Condición:} El catálogo de estados de un proyecto no tiene información.
	\begin{enumerate}
		\UCpaso[\UCsist] Muestra el mensaje \cdtIdRef{MSG12}{Ha ocurrido un error} en la pantalla \IUref{IU2}{Gestionar proyectos de Administrador} para indicar que no es posible realizar la operación debido a la falta de información necesaria para el sistema.
		\item[- -] - - {\em {Fin del caso de uso}}.%
	\end{enumerate}
	
	%--------------------------------------
	\hypertarget{CU2-2:TAB}{\textbf{Trayectoria alternativa B}}\\
	\noindent \textbf{Condición:} No hay ningún colaborador registrado.
	\begin{enumerate}
		\UCpaso[\UCsist] Muestra el mensaje \cdtIdRef{MSG12}{Ha ocurrido un error} en la pantalla \IUref{IU2}{Gestionar proyectos de Administrador} para indicar que no es posible realizar la operación debido a la falta de información necesaria para el sistema.
		\item[- -] - - {\em {Fin del caso de uso}}.%
	\end{enumerate}
	
	%--------------------------------------
	\hypertarget{CU2-2:TAC}{\textbf{Trayectoria alternativa C}}\\
	\noindent \textbf{Condición:} El actor desea cancelar la operación.
	\begin{enumerate}
		\UCpaso[\UCactor] Solicita cancelar la operación oprimiendo el botón \IUbutton{Cancelar} de la pantalla \IUref{IU2.1}{Registrar Proyecto}
		\UCpaso[\UCsist] Muestra la pantalla \IUref{IU2}{Gestionar proyectos de Administrador}.
		\item[- -] - - {\em {Fin del caso de uso}}.%
	\end{enumerate}
	%--------------------------------------	
	\hypertarget{CU2-2:TAD}{\textbf{Trayectoria alternativa D}}\\
	\noindent \textbf{Condición:} El actor no ingresó algún dato marcado como obligatorio.
	\begin{enumerate}
		\UCpaso[\UCsist] Muestra el mensaje \cdtIdRef{MSG4}{Dato obligatorio} señalando el campo que presenta el error en la pantalla \IUref{IU2.1}{Registrar Proyecto}.
		\UCpaso Regresa al paso \ref{CU2.1-P5} de la trayectoria principal.
		\item[- -] - - {\em {Fin de la trayectoria}}.%
	\end{enumerate}
	%--------------------------------------
	\hypertarget{CU2-2:TAE}{\textbf{Trayectoria alternativa E}}\\
	\noindent \textbf{Condición:} El actor ingresó un dato con un número de caracteres fuera del rango permitido.
	\begin{enumerate}
		\UCpaso[\UCsist] Muestra el mensaje \cdtIdRef{MSG6}{Longitud inválida} señalando el campo que presenta el error en la pantalla \IUref{IU2.1}{Registrar Proyecto}.
		\UCpaso Regresa al paso \ref{CU2.1-P5} de la trayectoria principal.
		\item[- -] - - {\em {Fin de la trayectoria}}.%
	\end{enumerate}
	%-------------------------------------
	\hypertarget{CU2-2:TAF}{\textbf{Trayectoria alternativa F}}\\
	\noindent \textbf{Condición:} El actor ingresó un dato con un formato o tipo de dato incorrecto.
	\begin{enumerate}
		\UCpaso[\UCsist] Muestra el mensaje \cdtIdRef{MSG29}{Formato incorrecto} señalando el campo que presenta el error en la pantalla \IUref{IU2.1}{Registrar Proyecto}.
		\UCpaso Regresa al paso \ref{CU2.1-P5} de la trayectoria principal.
		\item[- -] - - {\em {Fin de la trayectoria}}.
	\end{enumerate}
	%--------------------------------------
	\hypertarget{CU2-2:TAG}{\textbf{Trayectoria alternativa G}}\\
	\noindent \textbf{Condición:} El actor ingresó una clave de proyecto que ya existe dentro del sistema.
	\begin{enumerate}
		\UCpaso[\UCsist] Muestra el mensaje \cdtIdRef{MSG7}{Registro repetido} señalando el campo que presenta la duplicidad en la pantalla \IUref{IU2.1}{Registrar Proyecto}.
		\UCpaso Regresa al paso \ref{CU2.1-P5} de la trayectoria principal.
		\item[- -] - - {\em {Fin de la trayectoria}}.
	\end{enumerate}
	%--------------------------------------	
	\hypertarget{CU2-2:TAH}{\textbf{Trayectoria alternativa H}}\\
	\noindent \textbf{Condición:} El actor ingresó un nombre de proyecto que ya existe dentro del sistema.
	\begin{enumerate}
		\UCpaso[\UCsist] Muestra el mensaje \cdtIdRef{MSG7}{Registro repetido} señalando el campo que presenta la duplicidad en la pantalla \IUref{IU2.1}{Registrar Proyecto}.
		\UCpaso Regresa al paso \ref{CU2.1-P5} de la trayectoria principal.
		\item[- -] - - {\em {Fin de la trayectoria}}.
	\end{enumerate}
	%--------------------------------------
	\hypertarget{CU2-2:TAI}{\textbf{Trayectoria alternativa I}}\\
	\noindent \textbf{Condición:} La fecha de termino programada es menor a la fecha de inicio programada.
	\begin{enumerate}
		\UCpaso[\UCsist] Muestra el mensaje \cdtIdRef{MSG26}{Orden de fechas} en el campo de fecha de término programada en la pantalla \IUref{IU2.1}{Registrar Proyecto}.
		\UCpaso Regresa al paso \ref{CU2.1-P5} de la trayectoria principal.
		\item[- -] - - {\em {Fin de la trayectoria}}.
	\end{enumerate}