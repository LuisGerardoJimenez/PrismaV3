	\begin{UseCase}{CU2}{Gestionar proyectos de Administrador}{
		Este caso de uso permite al Administrador visualizar todos los proyectos registrados en el sistema, además sirve como punto de acceso para registrar, modificar o eliminar un proyecto.
	}
		\UCitem{Versión}{\color{Gray}0.1}
		\UCitem{Actor}{\hyperlink{admin}{Administrador}}
		\UCitem{Propósito}{Proporcionar al actor un mecanismo para llevar el control de los proyectos.}
		\UCitem{Entradas}{Ninguna}
		\UCitem{Salidas}{\begin{itemize}
				\item \hyperlink{proyectoEntidad}{Proyecto}: Tabla que muestra la \cdtRef{proyectoEntidad:claveProyecto}{clave}, \cdtRef{proyectoEntidad:nombreProyecto}{nombre} y el \cdtRef{proyectoEntidad:liderProyecto}{Líder de Proyecto} de todos los proyectos existentes.
		\end{itemize}}
		\UCitem{Destino}{Pantalla}
		\UCitem{Precondiciones}{Ninguna}
		\UCitem{Postcondiciones}{Ninguna}
		\UCitem{Errores}{\begin{itemize}
		\item \cdtIdRef{MSG2}{No existe información}: Se muestra en la pantalla \IUref{IU2}{Gestionar proyectos de Administrador} cuando no existen proyectos registrados
		\end{itemize}
		}
		\UCitem{Tipo}{Caso de uso primario}
	\end{UseCase}
%--------------------------------------
	\begin{UCtrayectoria}
		\UCpaso[\UCactor] Solicita gestionar los proyectos presionando la opción ''Proyectos'' del menú \IUref{MN1}{Menú de Administrador}.
		\UCpaso[\UCsist] Busca la información de los proyectos registrados. \hyperlink{CU2:TAA}{[Trayectoria A]}
		\UCpaso[\UCsist] Muestra la información de los proyectos en la pantalla \IUref{IU2}{Gestionar proyectos de Administrador}.
		\UCpaso[\UCactor] Gestiona los proyectos a través de los botones: \IUbutton{Registrar}, \editar  y \eliminar. \label{P4}
	\end{UCtrayectoria}		
%--------------------------------------
	\hypertarget{CU2:TAA}{\textbf{Trayectoria alternativa A}}\\
\noindent \textbf{Condición:} No existen registros de proyectos.
\begin{enumerate}
	\UCpaso[\UCsist] Muestra el mensaje \cdtIdRef{MSG2}{No existe información} en la pantalla \IUref{IU2}{Gestionar proyectos de Administrador} para indicar que no hay registros de proyectos para mostrar.
	\item[- -] - - {\em {Fin del caso de uso}}.%
\end{enumerate}
%--------------------------------------

\subsubsection{Puntos de extensión}

\UCExtenssionPoint{El actor requiere registrar un proyecto.}{Paso \ref{P4} de la trayectoria principal.}{\UCref{CU2.1}{Registrar Proyecto}}
\UCExtenssionPoint{El actor requiere modificar un proyecto.}{Paso \ref{P4} de la trayectoria principal.}{\UCref{CU2.2}{Modificar Proyecto}}
\UCExtenssionPoint{El actor requiere eliminar un proyecto.}{Paso \ref{P4} de la trayectoria principal.}{\UCref{CU2.3}{Eliminar Proyecto}}
