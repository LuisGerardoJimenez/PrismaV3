	\begin{UseCase}{CU9.2}{Modificar Mensaje}{
		Este caso de uso permite al analista modificar la información de un mensaje.
	}
		\UCitem{Versión}{\color{Gray}0.1}
		\UCitem{Actor}{\hyperlink{jefe}{Líder de Análisis}, \hyperlink{analista}{Analista}}
		\UCitem{Propósito}{Modificar la información de mensaje.}
		\UCitem{Entradas}{
		\begin{itemize}
			\item \cdtRef{MSGEntidad:numeroMSG}{Número:} Se escribe desde el teclado.
			\item \cdtRef{MSGEntidad:nombreMSG}{Nombre:} Se escribe desde el teclado.
			\item \cdtRef{MSGEntidad:descripcionMSG}{Nombre:} Se escribe desde el teclado.
			\item \cdtRef{MSGEntidad:redaccionMSG}{Redacción:} Se escribe desde el teclado.
			\item \cdtRef{MSGEntidad:paramtrizadoMSG}{Parametrizado:} Se escribe desde el teclado.
			\item Parámetros: Se escribe desde el teclado.
		\end{itemize}	
		}
		\UCitem{Salidas}{\begin{itemize}
				\item \cdtRef{MSGEntidad:numeroMSG}{Número:} Lo obtiene el sistema.
				\item \cdtRef{MSGEntidad:nombreMSG}{Nombre:} Lo obtiene el sistema.
				\item \cdtRef{MSGEntidad:descripcionMSG}{Nombre:} Lo obtiene el sistema.
				\item \cdtRef{MSGEntidad:redaccionMSG}{Redacción:} Lo obtiene el sistema.
				\item \cdtRef{MSGEntidad:paramtrizadoMSG}{Parametrizado:} Lo obtiene el sistema.
				\item Parámetros: Lo obtiene el sistema.
				\item \cdtIdRef{MSG1}{Operación exitosa}: Se muestra en la pantalla \IUref{IU10}{Gestionar Mensajes} para indicar que la modificación exitosa.
		\end{itemize}}
		\UCitem{Destino}{Pantalla}
		\UCitem{Precondiciones}{\begin{itemize}
				\item Que el mensaje no se encuentre asociado a un caso de uso en estado ''Liberado''.
		\end{itemize}}
		\UCitem{Postcondiciones}{Ninguna}
		\UCitem{Errores}{\begin{itemize}
		\item \cdtIdRef{MSG4}{Dato obligatorio}: Se muestra en la pantalla \IUref{IU10.2}{Modificar Mensaje} cuando no se ha ingresado un dato marcado como obligatorio.
		\item \cdtIdRef{MSG29}{Formato incorrecto}: Se muestra en la pantalla \IUref{IU10.2}{Modificar Mensaje} cuando el tipo de dato ingresado no cumple con el tipo de dato solicitado en el campo.
		\item \cdtIdRef{MSG5}{Formato de campo incorrecto}: Se muestra en la pantalla \IUref{IU10.2}{Modificar Mensaje} cuando el número de mensaje contiene un carácter no válido.
		\item \cdtIdRef{MSG6}{Longitud inválida}: Se muestra en la pantalla \IUref{IU10.2}{Modificar Mensaje} cuando se ha excedido la longitud de alguno de los campos.
		\item \cdtIdRef{MSG7}{Registro repetido}: Se muestra en la pantalla \IUref{IU10.2}{Modificar Mensaje} cuando se registre un mensaje con un nombre o número que ya se encuentre registrado en el sistema.
		\item \cdtIdRef{MSG18}{Caracteres inválidos}: Se muestra en la pantalla \IUref{IU10.2}{Modificar Mensaje} cuando el nombre demensaje contiene un carácter no válido
		\end{itemize}.
		}
		\UCitem{Tipo}{Secundario, extiende del caso de uso \UCref{CU9}{Gestionar Mensajes}.}
	\end{UseCase}
%--------------------------------------
	\begin{UCtrayectoria}
		\UCpaso[\UCactor] Solicita registrar una entidad oprimiendo el botón \editar de la pantalla \IUref{IU10}{Gestionar Mensajes}.
		\UCpaso[\UCsist] Obtiene la información del mensaje seleccionado.
		\UCpaso[\UCsist] Verifica que el mensaje pueda modificarse con base en la regla de negocio \BRref{RN5}{Modificación de elementos asociados a casos de uso liberados}. \Trayref{MMSG-I}
		\UCpaso[\UCsist] Verifica que el mensaje sea parametrizado.
		\UCpaso[\UCsist] Muestra la información encontrada en la pantalla \IUref{IU10.2A}{Modificar Mensaje: Parametrizado}. \Trayref{MMSG-A}
		\UCpaso[\UCactor] Modifica el número, nombre y descripción del mensaje. \label{CU9.2-P6}
		\UCpaso[\UCactor] Modifica descripción de los parámetros en la pantalla \IUref{IU10.2A}{Modificar Mensaje: Parametrizado}.
		\UCpaso[\UCactor] Solicita guardar los cambios del mensaje oprimiendo el botón \IUbutton{Aceptar} de la pantalla \IUref{IU10.2}{Modificar Mensaje}. \label{CU9.2-P8} \Trayref{MMSG-B} 
		\UCpaso[\UCsist] Verifica que el actor ingrese todos los campos obligatorios con base en la regla de negocio \BRref{RN8}{Datos obligatorios}. \Trayref{MMSG-C}
		\UCpaso[\UCsist] Verifica que los datos requeridos sean proporcionados correctamente con base en la regla de negocio \BRref{RN7}{Información correcta}. \Trayref{MMSG-D} \Trayref{MMSG-E}
		\UCpaso[\UCsist] Verifica que el número del mensaje sea proporcionado correctamente con base en la regla de negocio \BRref{RN7}{Información correcta}. \Trayref{RBR-F}
		\UCpaso[\UCsist] Verifica que el número del mensaje no se encuentre registrado en el sistema con base en la regla de negocio \BRref{RN1}{Unicidad de números}. \Trayref{MMSG-G}
		\UCpaso[\UCsist] Verifica que el nombre de la entidad no se encuentre registrado en el sistema con base en la regla de negocio \BRref{RN6}{Unicidad de nombres}. \Trayref{MMSG-H} 
		\UCpaso[\UCsist] Modifica la información del mensaje en el sistema.
		\UCpaso[\UCsist] Muestra el mensaje \cdtIdRef{MSG1}{Operación exitosa} en la pantalla \IUref{IU10}{Gestionar Mensajes} para indicar al actor que el registro se ha modificado exitosamente.
	\end{UCtrayectoria}		
%--------------------------------------
	
	\begin{UCtrayectoriaA}{MMSG-A}{El mensaje no es parametrizado.}
		\UCpaso[\UCsist] Muestra la pantalla \IUref{IU10.2}{Modificar Mensaje: No Parametrizado}.
		\UCpaso[\UCactor] Modifica el número, nombre, descripción y redacción del mensaje. \label{CU9.2-AP-2}
		\UCpaso[\UCsist] Continúa en el paso \ref{CU9.2-P8} de la trayectoria principal.
	\end{UCtrayectoriaA}


	\begin{UCtrayectoriaA}{MMSG-B}{El actor desea cancelar la operación.}
		\UCpaso[\UCactor] Solicita cancelar la operación oprimiendo el botón \IUbutton{Cancelar} de la pantalla \IUref{IU10.2}{Modificar Mensaje}.
		\UCpaso[\UCsist] Muestra la pantalla \IUref{IU10}{Gestionar Mensajes}.
	\end{UCtrayectoriaA}

	\begin{UCtrayectoriaA}{MMSG-C}{El actor no ingresó algún dato marcado como obligatorio.}
		\UCpaso[\UCsist] Muestra el mensaje \cdtIdRef{MSG4}{Dato obligatorio} y señala el campo que presenta el error en la pantalla \IUref{IU10.2}{Modificar Mensaje}, indicando al actor que el dato es obligatorio.
		\UCpaso Regresa al paso \ref{CU9.2-P6} de la trayectoria principal o al paso \ref{CU9.2-AP-2} de la trayectoria alternativa A.
	\end{UCtrayectoriaA}

	\begin{UCtrayectoriaA}{MMSG-D}{El actor proporciona un dato que excede la longitud máxima.}
		\UCpaso[\UCsist] Muestra el mensaje \cdtIdRef{MSG6}{Longitud inválida} y señala el campo que excede la longitud en la pantalla \IUref{IU10.2}{Modificar Mensaje}, para indicar que el dato excede el tamaño máximo permitido.
		\UCpaso Regresa al paso \ref{CU9.2-P6} de la trayectoria principal o al paso \ref{CU9.2-AP-2} de la trayectoria alternativa A.
	\end{UCtrayectoriaA}

	\begin{UCtrayectoriaA}{MMSG-E}{El actor ingresó un tipo de dato incorrecto.}
		\UCpaso[\UCsist] Muestra el mensaje \cdtIdRef{MSG29}{Formato incorrecto} y señala el campo que presenta el dato inválido en la pantalla \IUref{IU10.2}{Modificar Mensaje}, para indicar que se ha ingresado un tipo de dato inválido.
		\UCpaso Regresa al paso \ref{CU9.2-P6} de la trayectoria principal.
	\end{UCtrayectoriaA}

	\begin{UCtrayectoriaA}{MMSG-F}{El actor ingresó un número de mensaje con un tipo de dato incorrecto.}
		\UCpaso[\UCsist] Muestra el mensaje \cdtIdRef{MSG5}{Formato de campo incorrecto} y señala el campo que presenta el dato inválido en la pantalla \IUref{IU10.2}{Modificar Mensaje}, para indicar que se ha ingresado un tipo de dato inválido.
		\UCpaso Regresa al paso \ref{CU9.2-P6} de la trayectoria principal.
	\end{UCtrayectoriaA}

	\begin{UCtrayectoriaA}{MMSG-G}{El actor ingresó un número de mensaje repetido.}
		\UCpaso[\UCsist] Muestra el mensaje \cdtIdRef{MSG7}{Registro repetido} señala el campo que presenta la duplicidad en la pantalla \IUref{IU10.2}{Modificar Mensaje}, indicando al actor que existe un mensaje con el mismo número.
		\UCpaso Regresa al paso \ref{CU9.2-P6} de la trayectoria principal o al paso \ref{CU9.2-AP-2} de la trayectoria alternativa A.
	\end{UCtrayectoriaA}
	
	\begin{UCtrayectoriaA}{MMSG-H}{El actor ingresó un nombre de mensaje repetido.}
		\UCpaso[\UCsist] Muestra el mensaje \cdtIdRef{MSG7}{Registro repetido} y señala el campo que presenta la duplicidad en la pantalla \IUref{IU10.2}{Modificar Mensaje}, indicando al actor que existe un mensaje con el mismo nombre.
		\UCpaso Regresa al paso \ref{CU9.2-P6} de la trayectoria principal o al paso \ref{CU9.2-AP-2} de la trayectoria alternativa A.
	\end{UCtrayectoriaA}

	\begin{UCtrayectoriaA}{MMSG-I}{El mensaje no puede modificarse debido a que se encuentra asociado a casos de uso liberados.}
		\UCpaso[\UCsist] Oculta el botón \editar del mensaje que esta asociado a casos de uso liberados.
	\end{UCtrayectoriaA}
