	\begin{UseCase}{CU3.1}{Registrar persona}{
	Este caso de uso permite registrar la información de una persona que podrá ser elegida como colaborador de un proyecto.
	}
		\UCitem{Versión}{\color{Gray}0.1}
		\UCitem{Actor}{\hyperlink{admin}{Administrador}}
		\UCitem{Propósito}{Registrar la información de una persona.}
		\UCitem{Entradas}{
		\begin{itemize}
			\item \cdtRef{colaboradorEntidad:curpColaborador}{CURP}: Se escribe desde el teclado.
			\item \cdtRef{colaboradorEntidad:nombreColaborador}{Nombre}: Se escribe desde el teclado.
			\item \cdtRef{colaboradorEntidad:pApellidoColaborador}{Primer Apellido}: Se escribe desde el teclado.
			\item \cdtRef{colaboradorEntidad:sApellidoColaborador}{Segundo Apellido}: Se escribe desde el teclado.
			\item \cdtRef{colaboradorEntidad:correoColaborador}{Correo electrónico}: Se escribe desde el teclado.
			\item \cdtRef{colaboradorEntidad:passColaborador}{Contraseña}: Se escribe desde el teclado.
		\end{itemize}	
		}
		\UCitem{Salidas}{\begin{itemize}
				\item \cdtIdRef{MSG1}{Operación exitosa}: Se muestra en la pantalla \IUref{IU3}{Gestionar Personal} para indicar que el registro fue exitoso.
		\end{itemize}}
		\UCitem{Destino}{Pantalla}
		\UCitem{Precondiciones}{Ninguna}
		\UCitem{Postcondiciones}{
		\begin{itemize}
			\item Se registrará una persona en el sistema.
			\item La persona registrada se podrá elegir para participar en algún proyecto.
		\end{itemize}
		}
		\UCitem{Errores}{\begin{itemize}
		\item \cdtIdRef{MSG4}{Dato obligatorio}: Se muestra en la pantalla \IUref{IU3.1}{Registrar Persona} cuando no se ha ingresado un dato marcado como obligatorio.
		\item \cdtIdRef{MSG5}{Dato incorrecto}: Se muestra en la pantalla \IUref{IU3.1}{Registrar Persona} cuando el tipo de dato ingresado no cumple con el tipo de dato solicitado en
		el campo.
		\item \cdtIdRef{MSG6}{Longitud inválida}: Se muestra en la pantalla \IUref{IU3.1}{Registrar Persona} cuando se ha excedido la longitud de alguno de los campos.
		\item \cdtIdRef{MSG7}{Registro repetido}: Se muestra en la pantalla \IUref{IU3.1}{Registrar Persona} cuando se registre una persona con una CURP que ya se encuentra registrada en el sistema.
		\end{itemize}
		}
		\UCitem{Tipo}{Secundario, extiende del caso de uso \UCref{CU3}{Gestionar Personal}.}
	\end{UseCase}
%--------------------------------------
	\begin{UCtrayectoria}
		\UCpaso[\UCactor] Solicita registrar una persona oprimiendo el botón \IUbutton{Registrar} de la pantalla \IUref{IU3}{Gestionar Personal}.
		\UCpaso[\UCsist] Muestra la pantalla \IUref{IU3.1}{Registrar Persona}.
		\UCpaso[\UCactor] Ingresa la información solicitada en la pantalla. \label{CU3.1-P3}
		\UCpaso[\UCactor] Solicita guardar el proyecto oprimiendo el botón \IUbutton{Aceptar} de la pantalla \IUref{IU3.1}{Registrar Persona}. \Trayref{A}
		\UCpaso[\UCsist] Verifica que el actor ingrese todos los campos obligatorios con base en la regla de negocio \BRref{RN8}{Datos obligatorios}. \Trayref{B}
		\UCpaso[\UCsist] Verifica que la CURP de persona no se encuentre registrado en el sistema con base en la regla de negocio \BRref{RN6}{Unicidad de nombres}. \Trayref{C}
		\UCpaso[\UCsist] Verifica que los datos requeridos sean proporcionados correctamente con base en la regla de negocio \BRref{RN7}{Información correcta}. \Trayref{D} \Trayref{E}
		\UCpaso[\UCsist] Registra la información de la persona en el sistema
		\UCpaso[\UCsist] Envía un correo con el mensaje \cdtIdRef{MSG25}{Datos de sesión} a la cuenta de correo electrónico proporcionada por el actor.
		\UCpaso[\UCsist] Muestra el mensaje \cdtIdRef{MSG1}{Operación exitosa} en la pantalla \IUref{IU3}{Gestionar Personal} para indicar al actor que el registro se ha realizado exitosamente.
	\end{UCtrayectoria}		
%--------------------------------------
		\begin{UCtrayectoriaA}{A}{El actor desea cancelar la operación.}
			\UCpaso[\UCactor] Solicita cancelar la operación oprimiendo el botón \IUbutton{Cancelar} de la pantalla \IUref{IU3.1}{Registrar Persona}.
			\UCpaso[\UCsist] Muestra la pantalla \IUref{IU3}{Gestionar Personal}
		\end{UCtrayectoriaA}

%--------------------------------------

		\begin{UCtrayectoriaA}{B}{No hay ningún colaborador registrado.}
	\UCpaso[\UCsist] Muestra el mensaje \cdtIdRef{MSG17}{Falta de información} en la pantalla \IUref{IU2}{Gestionar proyectos de Administrador} indicando que no hay colaboradores registrados.
		\end{UCtrayectoriaA}

	\begin{UCtrayectoriaA}{B}{El actor no ingresó algún dato marcado como obligatorio.}
		\UCpaso[\UCsist] Muestra el mensaje \cdtIdRef{MSG4}{Dato obligatorio} y señala el campo que presenta el error en la pantalla \IUref{IU3.1}{Registrar Personal}, indicando al actor que el dato es obligatorio.
		\UCpaso Regresa al paso \ref{CU3.1-P3} de la trayectoria principal.
	\end{UCtrayectoriaA}
	
	\begin{UCtrayectoriaA}{C}{El actor ingresó una CURP repetida.}
		\UCpaso[\UCsist] Muestra el mensaje \cdtIdRef{MSG7}{Registro repetido} y señala el campo que presenta la duplicidad en la pantalla \IUref{IU3.1}{Registrar Personal}, indicando al actor que existe una persona con la misma CURP.
		\UCpaso Regresa al paso \ref{CU3.1-P3} de la trayectoria principal.
	\end{UCtrayectoriaA}

	\begin{UCtrayectoriaA}{D}{El actor proporciona un dato que excede la longitud máxima.}
		\UCpaso[\UCsist] Muestra el mensaje \cdtIdRef{MSG6}{Longitud inválida} y señala el campo que excede la longitud en la pantalla \IUref{IU3.1}{Registrar Personal}, para indicar que el dato excede el tamaño máximo permitido.
		\UCpaso Regresa al paso \ref{CU3.1-P3} de la trayectoria principal.
	\end{UCtrayectoriaA}

	\begin{UCtrayectoriaA}{E}{El actor ingresó un tipo de dato incorrecto.}
		\UCpaso[\UCsist] Muestra el mensaje \cdtIdRef{MSG5}{Dato incorrecto} y señala el campo que presenta el dato inválido en la pantalla \IUref{IU3.1}{Registrar Persona}, para indicar que se ha ingresado un tipo de dato inválido.
		\UCpaso Regresa al paso \ref{CU3.1-P3} de la trayectoria principal.
	\end{UCtrayectoriaA}

