	\begin{UseCase}{CU10.1}{Registrar Actor}{
		Un actor es un rol que va a interactuar con el sistema. Este caso de uso permite al analista registrar un actor.
	}
		\UCitem{Versión}{\color{Gray}0.1}
		\UCitem{Actor}{\hyperlink{jefe}{Líder de Análisis}, \hyperlink{analista}{Analista}}
		\UCitem{Propósito}{Registrar la información de actor.}
		\UCitem{Entradas}{
		\begin{itemize}
			\item \cdtRef{actorEntidad:nombreACT}{Nombre:} Se escribe desde el teclado.
			\item \cdtRef{actorEntidad:descripcionACT}{Descripción:} Se escribe desde el teclado.
			\item \cdtRef{actorEntidad:oCardinalidadACT}{Cardinalidad:} Se selecciona de una lista.
		\end{itemize}	
		}
		\UCitem{Salidas}{\begin{itemize}
				\item \cdtRef{actorEntidad:claveACT}{Clave:} Lo calcula el sistema mediante la regla de negocio \BRref{RN12}{Identifcador de elemento}.
				\item \cdtIdRef{MSG1}{Operación exitosa}: Se muestra en la pantalla \IUref{IU8}{Gestionar Actores} para indicar que el registro fue exitoso.
		\end{itemize}}
		\UCitem{Destino}{Pantalla}
		\UCitem{Precondiciones}{Ninguna}
		\UCitem{Postcondiciones}{
		\begin{itemize}
			\item Se registrará un actor en el sistema.
			\item El actor podrá ser referenciado en casos de uso.
		\end{itemize}
		}
		\UCitem{Errores}{\begin{itemize}
		\item \cdtIdRef{MSG4}{Dato obligatorio}: Se muestra en la pantalla \IUref{IU8.1}{Registrar Actor} cuando no se ha ingresado un dato marcado como obligatorio.
		\item \cdtIdRef{MSG29}{Formato incorrecto}: Se muestra en la pantalla \IUref{IU8.1}{Registrar Actor} cuando el tipo de dato ingresado no cumple con el tipo de dato solicitado en el campo.
		\item \cdtIdRef{MSG6}{Longitud inválida}: Se muestra en la pantalla \IUref{IU8.1}{Registrar Actor} cuando se ha excedido la longitud de alguno de los campos.
		\item \cdtIdRef{MSG7}{Registro repetido}: Se muestra en la pantalla \IUref{IU8.1}{Registrar Actor} cuando se registre un actor con un nombre que ya se encuentre registrado en el sistema.
		\item \cdtIdRef{MSG12}{Ha ocurrido un error}: Se muestra en la pantalla \IUref{IU8}{Gestionar Actores} cuando no existe información base para el sistema.
		\end{itemize}.
		}
		\UCitem{Tipo}{Secundario, extiende del caso de uso \UCref{CU10}{Gestionar Actores}.}
	\end{UseCase}
%--------------------------------------
	\begin{UCtrayectoria}
		\UCpaso[\UCactor] Solicita registrar una actor oprimiendo el botón \IUbutton{Registrar} de la pantalla \IUref{IU8}{Gestionar Actores}.
		\UCpaso[\UCactor] Verifica que exista información referente a la Cardinalidad con base en la regla de negocio \BRref{RN20}{Verificación de catálogos}. \Trayref{RACT-A}
		\UCpaso[\UCsist] Muestra la pantalla \IUref{IU8.1}{Registrar Actor}.
		\UCpaso[\UCactor] Ingresa la información solicitada en la pantalla. \label{CU10.1-P4}
		\UCpaso[\UCactor] Solicita guardar la información del actor oprimiendo el botón \IUbutton{Aceptar} de la pantalla \IUref{IU8.1}{Registrar Actor}. \label{CU10.1-P5} \Trayref{RACT-B} 
		\UCpaso[\UCsist] Verifica que el actor ingrese todos los campos obligatorios con base en la regla de negocio \BRref{RN8}{Datos obligatorios}. \Trayref{RACT-C}
		\UCpaso[\UCsist] Verifica que los datos requeridos sean proporcionados correctamente con base en la regla de negocio \BRref{RN7}{Información correcta}. \Trayref{RACT-D} \Trayref{RACT-E}
		\UCpaso[\UCsist] Verifica que el nombre del actor no se encuentre registrado en el sistema con base en la regla de negocio \BRref{RN6}{Unicidad de nombres}. \Trayref{RACT-F} 
		\UCpaso[\UCsist] Registra la información del actor en el sistema.
		\UCpaso[\UCsist] Muestra el mensaje \cdtIdRef{MSG1}{Operación exitosa} en la pantalla \IUref{IU8}{Gestionar Actores} para indicar al actor que el registro se ha realizado exitosamente.
	\end{UCtrayectoria}		
%--------------------------------------
	
	\begin{UCtrayectoriaA}{RACT-A}{No existe información en los catálogos.}
		\UCpaso[\UCactor] Muestra el mensaje \cdtIdRef{MSG12}{Ha ocurrido un error} en la pantalla \IUref{IU8}{Gestionar Actores}.
	\end{UCtrayectoriaA}

	\begin{UCtrayectoriaA}{RACT-B}{El actor desea cancelar la operación.}
		\UCpaso[\UCactor] Solicita cancelar la operación oprimiendo el botón \IUbutton{Cancelar} de la pantalla \IUref{IU8.1}{Registrar Actor}.
		\UCpaso[\UCsist] Muestra la pantalla \IUref{IU8}{Gestionar Actores}.
	\end{UCtrayectoriaA}

	\begin{UCtrayectoriaA}{RACT-C}{El actor no ingresó algún dato marcado como obligatorio.}
		\UCpaso[\UCsist] Muestra el mensaje \cdtIdRef{MSG4}{Dato obligatorio} y señala el campo que presenta el error en la pantalla \IUref{IU8.1}{Registrar Actor}, indicando al actor que el dato es obligatorio.
		\UCpaso Regresa al paso \ref{CU10.1-P4} de la trayectoria principal.
	\end{UCtrayectoriaA}

	\begin{UCtrayectoriaA}{RACT-D}{El actor proporciona un dato que excede la longitud máxima.}
		\UCpaso[\UCsist] Muestra el mensaje \cdtIdRef{MSG6}{Longitud inválida} y señala el campo que excede la longitud en la pantalla \IUref{IU8.1}{Registrar Actor}, para indicar que el dato excede el tamaño máximo permitido.
		\UCpaso Regresa al paso \ref{CU10.1-P4} de la trayectoria principal.
	\end{UCtrayectoriaA}

	\begin{UCtrayectoriaA}{RACT-E}{El actor ingresó un tipo de dato incorrecto.}
		\UCpaso[\UCsist] Muestra el mensaje \cdtIdRef{MSG29}{Formato incorrecto} y señala el campo que presenta el dato inválido en la pantalla \IUref{IU8.1}{Registrar Actor}, para indicar que se ha ingresado un tipo de dato inválido.
		\UCpaso Regresa al paso \ref{CU10.1-P4} de la trayectoria principal.
	\end{UCtrayectoriaA}
	
	\begin{UCtrayectoriaA}{RACT-F}{El actor ingresó un nombre de un actor repetido.}
		\UCpaso[\UCsist] Muestra el mensaje \cdtIdRef{MSG7}{Registro repetido} y señala el campo que presenta la duplicidad en la pantalla \IUref{IU8.1}{Registrar Actor}, indicando al actor que existe un actor con el mismo nombre.
		\UCpaso Regresa al paso \ref{CU10.1-P4} de la trayectoria principal.
	\end{UCtrayectoriaA}
