\begin{BusinessEntity}{MSGEntidad}{Mensaje}
		
	\Battr{claveMSG}{Clave: }{Clave que permitirá distinguir que el Elemento es un Mensaje. Es una palabra corta y este dato es \hyperlink{tRequerido}{requerido} {\em (no se puede omitir)}.}

	\Battr{numeroMSG}{Número: }{Número de Mensaje. Es un valor numérico entero y este dato es \hyperlink{tRequerido}{requerido} {\em (no se puede omitir)}.}
	
	\Battr{nombreMSG}{Nombre: }{Nombre que identificará al Mensaje. Es una frase o enunciado y este dato es \hyperlink{tRequerido}{requerido} {\em (no se puede omitir)}.}
	
	\Battr{descripcionMSG}{Descripción: }{Texto que describirá al mensaje. Descrita en uno o más párrafos y este dato es \hyperlink{tRequerido}{requerido} {\em (no se puede omitir)}.}
	
	\Battr{redaccionMSG}{Redacción: }{Redacción del Mensaje. Descrita en uno o más párrafos y este dato es \hyperlink{tRequerido}{requerido} {\em (no se puede omitir)}.}
	
	\Battr{paramMSG}{Parametrizado: }{Bandera que indicará si el mensaje se encuentra parametrizado. Indica ''si'' o ''no''. y este dato es \hyperlink{tRequerido}{requerido} {\em (no se puede omitir)}.}
\end{BusinessEntity}

\subsubsection{Relaciones}
\begin{BusinessFact}{vPAramRelMSG}{Valor del parámetro en Mensaje}
	\BRitem{\textbf{Descripción: }}{Valor de los parámetros ocupados al definir el Mensaje como parametrizado.}
	\BRitem{\textbf{Tipo: }}{\relAgregacion}
	\BRitem{\textbf{Cardinalidad: }}{Uno a Muchos}
\end{BusinessFact}
